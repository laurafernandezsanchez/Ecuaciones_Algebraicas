\section{La correspondencia de Galois} 
\begin{definicion}{Homomorfismos de extensiones}
    Dadas $K \subseteq L_1, L_2$ extensiones.

    Si $L_1$ y $L_2$ son dos extensiones de $K$, entonces un \textbf{homomorfismo} de $L_1/K$ en $L_2/K$ (también llamado $K$-homomorfismo) es un homomorfismo de cuerpos $f : L_1 \to L_2$ tal que $f(a) = a$ para todo $a \in K$.
    
    Un \textbf{endomorfismo} de una extensión $L/K$ es un homomorfismo de $L/K$ en sí misma. Un \textbf{isomorfismo} de extensiones (o $K$-isomorfismo) es un homomorfismo de extensiones que es isomorfismo de cuerpos y un \textbf{automorfismo} de extensiones (o $K$-automorfismo) es un isomorfismo de una extensión de $K$ en sí misma.
\end{definicion}

\begin{definicion}{Grupo de Galois}
    El \textbf{grupo de Galois} de $L/K$ es el conjunto de $K$-automorfismos de $L/K$ con la composición de aplicaciones. 
    
    Lo denotamos por $\operatorname{Gal}(L/K)$.
\end{definicion}

\begin{definicion}{Subextensión}
    Una \textbf{subextensión} de $L/K$ es un cuerpo $M$ tal que $K \subseteq M \subseteq L$.\\
    Recordemos que $\operatorname{Sub}(L/K)$ denota el conjunto de las subextensiones de $L/K$. 
\end{definicion}

\begin{definicion}{Extensiones admisibles}
    Dos extensiones $L_1/K$ y $L_2/K$ son \textbf{admisibles} si existe un cuerpo $L$ tal que $L_1 \subseteq L$ y $L_2 \subseteq L$.
\end{definicion}

\begin{observacion}{Convenios y propiedades básicas}
    En todo momento supondremos $1 \neq 0 \implies$ Todos los homomorfismos entre cuerpos son inyectivos.
    Además, los $K$-homomorfismos son homomorfismos de $K$-espacios vectoriales.
    
    De esta forma siempre que exista un homomorfismo de cuerpos $f : K \to L$, el cuerpo $L$ contiene un subcuerpo isomorfo a $K$, la imagen $f(K)$ de $f$. Por otro lado $K$ admite una extensión isomorfa a $L$, a saber el conjunto $K \cup (L \setminus f(K))$, en el que se define el producto de la forma obvia. 
    
    Abusaremos a menudo de la notación y cada vez que tengamos un homomorfismo de cuerpos $f : K \to L$, simplemente consideraremos $K$ como subcuerpo de $L$, identificando los elementos de $K$ y $f(K)$, a través de $f$.
    
    De aquí tenemos que, si $f: K \to L$ es un homomorfismo de cuerpos:
    \begin{observacion}{}
        \begin{enumerate}
            \item $f(K) \simeq K$ (como cuerpos). A veces simplemente escribiremos $K$ para referirnos a $f(K)$.
            \item Por convenio, $K$ es subcuerpo de $L$ identificando $K$ con $f(K)$.
        \end{enumerate}
    \end{observacion}
\end{observacion}

\begin{lema}{Propiedades de los homomorfismos sobre raíces}
    \begin{enumerate}
        \item Sean $\sigma : E \to L$ un homomorfismo de cuerpos y $p \in E[X]$. Si $\alpha$ es una raíz de $p$ en $E$, entonces $\sigma(\alpha)$ es una raíz de $\sigma(p)$. 
        \begin{itemize}
            \item \textit{Nota:} Esto asegura que la propiedad de ser algebraicamente cerrado es invariante bajo isomorfismos. 
        \end{itemize}
        
        \item Si $E/K$ y $L/K$ son extensiones de un cuerpo $K$, $p \in K[X]$ y $\sigma$ es un $K$-homomorfismo, entonces $\sigma$ se restringe a una aplicación inyectiva del conjunto de las raíces de $p$ en $E$ al conjunto de las raíces de $p$ en $L$. 
        
        \item En particular, si $E = L$ (es decir, si $\sigma \in \operatorname{Gal}(L/K)$), entonces esta restricción de $\sigma$ es una permutación del conjunto de las raíces de $p$ en $L$. 
    \end{enumerate}
\end{lema}

\begin{ejemplo}{Algunas extensiones con grupo trivial}
    Claramente $\operatorname{Gal}(K/K)=1$, pero no son éstas las únicas extensiones con grupo de Galois trivial. %
    \\Por ejemplo, si $a$ es un número racional positivo que no es el cubo de un número racional, entonces $p=X^{3}-a$ es irreducible en $\mathbb{Q}[X]$. %
    Las raíces de $p$ son $\alpha=\sqrt[3]{a}$, $\omega\alpha$ y $\omega^{2}\alpha$, donde $\omega$ es una raíz tercera primitiva de la unidad. %
    Como $\omega$ no es un número real, la única raíz de $p$ que pertenece a $\mathbb{Q}(\alpha)$ es $\alpha$ y por tanto $\operatorname{Gal}(\mathbb{Q}(\alpha)/\mathbb{Q})=1$ (¿por qué?). %
\end{ejemplo}

\begin{observacion}{Justificación del Ejemplo 1}
    Cualquier $K$-automorfismo $\sigma \in \operatorname{Gal}(K(\alpha)/K)$ está completamente determinado por la imagen del generador de la extensión, es decir, por $\sigma(\alpha)$. %
    
    
    Además, los homomorfismos de cuerpos preservan las raíces de los polinomios con coeficientes en el cuerpo base. Como $\alpha$ es raíz de $p \in K[X]$, su imagen $\sigma(\alpha)$ debe ser obligatoriamente otra raíz de $p$. %
    
    Por tanto, las únicas opciones teóricas son $\sigma(\alpha) \in \{\alpha, \omega\alpha, \omega^2\alpha\}$. %
    Sin embargo, $\sigma$ es un endomorfismo de $K(\alpha)$, lo que exige que $\sigma(\alpha) \in K(\alpha)$. Dado que $a > 0$ es racional, podemos considerar $K(\alpha) \subset \mathbb{R}$. Como $\omega = -\frac{1}{2} + i\frac{\sqrt{3}}{2} \notin \mathbb{R}$, se sigue que $\omega\alpha \notin \mathbb{R}$ y $\omega^2\alpha \notin \mathbb{R}$, por lo que ninguna de estas dos raíces pertenece a $K(\alpha)$. %
    La única asignación bien definida y posible es $\sigma(\alpha) = \alpha$, lo que implica que $\sigma$ es la aplicación identidad. %
\end{observacion}

\begin{ejemplo}{Extensiones de grado 2}
Si $L/K$ es una extensión de grado 2 y $\operatorname{car}(K) \neq 2$, entonces $|\operatorname{Gal}(L/K)| = 2$. % 

\textbf{¿Por qué ocurre esto?}
Si $\alpha\in L\setminus K$, entonces $L=K(\alpha)$ y por tanto $p=\operatorname{Min}_{K}(\alpha)$ tiene grado 2. Pongamos $p=X^{2}+aX+b=\left(X+\frac{a}{2}\right)^{2}+b-\frac{a^2}{4}$  

Para simplificar el estudio de la extensión, realizamos un cambio de variable para eliminar el término en $X$:
\[
p(X) = X^2 + aX + b = \left( X + \frac{a}{2} \right)^2 + \underbrace{b - \frac{a^2}{4}}_{-c}
\]
Definimos un nuevo generador $\beta = \alpha + \frac{a}{2}$. Como $\frac{a}{2} \in K$, se tiene que $K(\alpha) = K(\beta)$. % 
El polinomio mínimo de $\beta$ es ahora mucho más sencillo: $q(X) = X^2 - c$. Sus raíces son simplemente $\pm\beta$. % 

\textbf{¿Por qué el grupo de Galois tiene orden 2?}
Cualquier $K$-automorfismo $\sigma$ debe enviar una raíz de $q(X)$ a otra raíz de $q(X)$. % 
\begin{itemize}
    \item \textbf{Opción 1:} $\sigma(\beta) = \beta$. Esto define la aplicación identidad $id_L$. % 
    \item \textbf{Opción 2:} $\sigma(\beta) = -\beta$. Esto define un automorfismo no trivial (análogo a la conjugación compleja). % 
\end{itemize}
\end{ejemplo}

\begin{observacion}{Justificación del Ejemplo 2}
    Para asegurar que efectivamente tiene \textit{exactamente} dos elementos, debemos garantizar que las dos opciones teóricas ($\sigma(\beta) = \beta$ y $\sigma(\beta) = -\beta$) generan automorfismos distintos y bien definidos. %
    
    Primero, comprobamos que son distintos: como $\operatorname{car}(K) \neq 2$, se cumple que $2\beta \neq 0$ (ya que $\beta \notin K \implies \beta \neq 0$), por lo que $\beta \neq -\beta$. %
    Segundo, comprobamos que ambas opciones dan lugar a automorfismos válidos: $L$ es el cuerpo de escisión del polinomio $X^2-c$ sobre $K$. Al ser irreducible y tener raíces simples (es separable por ser de característica distinta de 2), la teoría elemental de extensiones asegura que por cada raíz en el cuerpo de escisión existe un $K$-automorfismo que envía el generador a dicha raíz. Por consiguiente, existe el automorfismo identidad ($\beta \mapsto \beta$) y un automorfismo no trivial ($\beta \mapsto -\beta$), formando un grupo de Galois de orden 2. %
\end{observacion}

\begin{ejemplo}{Automorfismos en extensiones de $\mathbb{R}$}
    Como un automorfismo de $\mathbb{R}$ ha de ser una aplicación creciente (¿por qué?), necesariamente $\operatorname{Gal}(\mathbb{R}/\mathbb{Q})=1$ y por tanto $\operatorname{Gal}(\mathbb{R}/K)=1$ para todo subcuerpo $K$ de $\mathbb{R}$ (¿por qué?). %
    De hecho, el único automorfismo de $\mathbb{R}$ es la identidad (¿por qué?). %
\end{ejemplo}

\begin{observacion}{Justificación del Ejemplo 3}
    Vamos a responder a las tres preguntas secuencialmente: %
    \begin{enumerate}
        \item \textbf{¿Por qué ha de ser creciente?} Sea $\sigma \in \operatorname{Aut}(\mathbb{R})$. Todo número real positivo $x > 0$ admite una raíz cuadrada real, es decir, $x = (\sqrt{x})^2$. Al aplicar $\sigma$, obtenemos $\sigma(x) = \sigma((\sqrt{x})^2) = (\sigma(\sqrt{x}))^2$. Un cuadrado en $\mathbb{R}$ es siempre positivo o cero. Como $\sigma$ es un automorfismo (y por tanto inyectivo, enviando $0 \mapsto 0$), para $x > 0$ se tiene estrictamente que $\sigma(x) > 0$. %
        Si tomamos $a < b$, entonces $b - a > 0$. Por la propiedad anterior, $\sigma(b-a) > 0$, y por linealidad $\sigma(b) - \sigma(a) > 0$, lo que implica $\sigma(a) < \sigma(b)$. Por lo tanto, $\sigma$ preserva el orden estrictamente.
        
        \item \textbf{¿Por qué el único automorfismo es la identidad?} Sabemos que todo automorfismo fija el cuerpo primo; por tanto, $\sigma(q) = q$ para todo $q \in \mathbb{Q}$. %
        Sea $x \in \mathbb{R}$ un número irracional. Por la densidad de $\mathbb{Q}$ en $\mathbb{R}$, podemos acotarlo entre dos sucesiones de racionales tales que $q_1 < x < q_2$. Dado que $\sigma$ es creciente (como demostramos arriba), preserva estas desigualdades: $\sigma(q_1) < \sigma(x) < \sigma(q_2)$. %
        Como $\sigma$ fija los racionales, esto se traduce en $q_1 < \sigma(x) < q_2$. Al tomar el límite cuando $q_1$ y $q_2$ tienden a $x$, por el Teorema del Sandwich (o encaje de intervalos), obtenemos forzosamente que $\sigma(x) = x$. Por tanto, $\sigma$ es la identidad en todo $\mathbb{R}$.
        
        \item \textbf{¿Por qué $\operatorname{Gal}(\mathbb{R}/K)=1$?} El grupo $\operatorname{Gal}(\mathbb{R}/K)$ está formado por los automorfismos de $\mathbb{R}$ que fijan $K$. Pero acabamos de demostrar que el único automorfismo general de $\mathbb{R}$ (es decir, el único elemento de $\operatorname{Aut}(\mathbb{R})$) es la identidad. Por tanto, exijamos que fije $K$ o que fije $\mathbb{Q}$, el único candidato posible sigue siendo la aplicación identidad. %
    \end{enumerate}
\end{observacion}

\begin{ejemplo}{}
    Sean $K=\mathbb{Q}(\sqrt{2},\sqrt{3})$ y $\sigma\in \operatorname{Gal}(K/\mathbb{Q})$. Entonces $\sigma(\sqrt{2})=\pm\sqrt{2}$ y $\sigma(\sqrt{3})=\pm\sqrt{3}$ y por tanto $\operatorname{Gal}(K/\mathbb{Q})$ tiene a lo sumo 4 elementos. %
    De hecho $\operatorname{Gal}(K/\mathbb{Q})$ tiene exactamente cuatro elementos. En efecto, en el Ejemplo (2) hemos visto que $\operatorname{Gal}(\mathbb{Q}(\sqrt{2})/\mathbb{Q})$ tiene 2 elementos. %
    Por otro lado $\sqrt{3}\notin\mathbb{Q}(\sqrt{2})$. %
    Por tanto, $K/\mathbb{Q}(\sqrt{2})$ es una extensión separable (¿por qué?) de grado 2, con lo que cada uno de los dos elementos de $\operatorname{Gal}(\mathbb{Q}(\sqrt{2})/\mathbb{Q})$ tiene dos extensiones a un homomorfismo de $K$ en una clausura algebraica de $K$ que, como además $K/\mathbb{Q}$ es normal (¿por qué?), estas dos extensiones son elementos de $\operatorname{Gal}(K/\mathbb{Q})$. %
    Por tanto $\operatorname{Gal}(K/\mathbb{Q})$ tiene cuatro elementos: $\sigma_{++}, \sigma_{+-}, \sigma_{-+}, \sigma_{--}$ dados por $\sigma_{ab}(\sqrt{2})=a\sqrt{2}$ y $\sigma_{ab}(\sqrt{3})=b\sqrt{3}$. %
\end{ejemplo}

\begin{observacion}{Justificación del Ejemplo 4}
    \begin{enumerate}
        \item \textbf{¿Por qué $K/\mathbb{Q}(\sqrt{2})$ es separable?} Esta extensión se obtiene adjuntando $\sqrt{3}$, cuyo polinomio mínimo sobre $\mathbb{Q}(\sqrt{2})$ es divisor de $X^2-3$. Estamos trabajando sobre una extensión de $\mathbb{Q}$, lo que significa que el cuerpo base tiene característica cero. Todo polinomio irreducible sobre un cuerpo de característica cero es separable (sus derivadas formales nunca son nulas), por lo que toda extensión algebraica en característica cero es automáticamente separable. %
        \item \textbf{¿Por qué $K/\mathbb{Q}$ es normal?} Una extensión finita es normal si y solo si es el cuerpo de escisión de algún polinomio sobre el cuerpo base. En este caso, $K = \mathbb{Q}(\sqrt{2},\sqrt{3})$ contiene todas las raíces del polinomio $f(X) = (X^2-2)(X^2-3) \in \mathbb{Q}[X]$, y de hecho es generado por ellas. Al ser cuerpo de escisión, la extensión es normal. %
    \end{enumerate}
\end{observacion}

\begin{ejemplo}{}
    Sea $\xi$ una raíz $n$-ésima primitiva de la unidad y sea $L=K(\xi)/K$ una extensión ciclotómica. %
    Si $\sigma\in \operatorname{Gal}(L/K)$, entonces $\sigma(\xi)=\xi^{i}$ para algún entero $i$ coprimo con $n$, y $\sigma$ está completamente determinada por el resto de $i$ módulo $n$. %
    Por tanto, tenemos una aplicación $\psi:\operatorname{Gal}(L/K)\rightarrow\mathbb{Z}_{n}^{*}$ que asocia $\sigma\in \operatorname{Gal}(L/K)$ con la única clase en $\mathbb{Z}_{n}^{*}$ que contiene a $i$ (con $\sigma(\xi)=\xi^{i}$). %
    Entonces $\psi$ es un homomorfismo inyectivo de grupos (comprobarlo) y por tanto $\operatorname{Gal}(L/K)$ es isomorfo a un subgrupo de $\mathbb{Z}_{n}^{*}$. En particular, el grupo de Galois de toda extensión ciclotómica es abeliano. %
    Si además $K=\mathbb{Q}$, entonces $\operatorname{Min}_{\mathbb{Q}}(\xi)=\Phi_{n}$, el $n$-ésimo polinomio ciclotómico (Teorema 3.9). %
    Por tanto, para cada $i$ coprimo con $n$ existe un elemento $\sigma\in \operatorname{Gal}(L=\mathbb{Q}(\xi)/\mathbb{Q})$ con $\sigma(\xi)=\xi^{i}$. En otras palabras, $\operatorname{Gal}(\mathbb{Q}(\xi)/\mathbb{Q})$ es isomorfo a $\mathbb{Z}_{n}^{*}$ y un isomorfismo $\tau:\mathbb{Z}_{n}^{*}\rightarrow \operatorname{Gal}(\mathbb{Q}(\xi)/\mathbb{Q})$ viene dado asociando $i\in\mathbb{Z}_{n}^{*}$ con el único automorfismo $\tau_{i}$ de $\mathbb{Q}(\xi)$ tal que $\tau_{i}(\xi)=\xi^{i}$. %
\end{ejemplo}

\begin{observacion}{Justificación del Ejemplo 5}
    \textbf{Comprobación de que $\psi$ es un homomorfismo inyectivo:} %
    
    Primero, veamos que respeta la operación del grupo (la composición). Sean $\sigma, \tau \in \operatorname{Gal}(L/K)$. Supongamos que $\psi(\sigma) = [i]$ y $\psi(\tau) = [j]$, lo que significa por definición que $\sigma(\xi) = \xi^i$ y $\tau(\xi) = \xi^j$. %
    Calculemos la imagen de $\xi$ bajo la composición $\sigma \circ \tau$:
    $$(\sigma \circ \tau)(\xi) = \sigma(\tau(\xi)) = \sigma(\xi^j)$$
    Como $\sigma$ es un homomorfismo de cuerpos, preserva los exponentes:
    $$\sigma(\xi^j) = (\sigma(\xi))^j = (\xi^i)^j = \xi^{ij}$$
    Esto nos dice que el automorfismo composición $\sigma \circ \tau$ eleva $\xi$ a la potencia $ij$. Por tanto, $\psi(\sigma \circ \tau) = [ij] = [i][j] = \psi(\sigma)\psi(\tau)$ en el grupo multiplicativo $\mathbb{Z}_{n}^{*}$. Esto demuestra que $\psi$ es un homomorfismo. %
    
    Segundo, evaluemos la inyectividad estudiando el núcleo. Supongamos que $\sigma \in \ker(\psi)$. Esto significa que $\psi(\sigma)$ es el elemento neutro de $\mathbb{Z}_{n}^{*}$, es decir, la clase $[1]$. Entonces, $\sigma(\xi) = \xi^1 = \xi$. %
    Dado que el cuerpo $L = K(\xi)$ está generado en su totalidad por $\xi$ y los elementos de $K$ (los cuales todo automorfismo de Galois deja fijos por definición), si un automorfismo deja fijo al generador $\xi$, obligatoriamente deja fijo a todo elemento de $L$. Luego $\sigma = \operatorname{id}_L$. Al ser su núcleo trivial, $\psi$ es inyectiva. %
\end{observacion}


%%%%%%% FIN SECCIÓN EJEMPLOS %%%%%%%

\begin{observacion}{Isomorfismos de grupos de Galois}
    Obsérvese que si $\phi:L\rightarrow L^{*}$ es un $K$-isomorfismo, entonces la aplicación $\operatorname{Gal}(L/K)\rightarrow \operatorname{Gal}(L^{*}/K)$ dada por $\sigma\mapsto\phi\sigma\phi^{-1}$ es un isomorfismo. 
\end{observacion}

Si $L/K$ es una extensión algebraica y $\overline{L}$ una clausura algebraica de $L$, entonces podemos ver cada elemento de $\operatorname{Gal}(L/K)$ como un elemento de $S_{1}^{L}=\{\sigma:L\rightarrow\overline{L} \mid \sigma|_{K}=1_{K}\}$.
Por tanto de la Proposición 4.9 deducimos: % 32]

\begin{proposicion}{}
    Si $L/K$ es una extensión finita entonces $|\operatorname{Gal}(L/K)|\le[L:K]_{s}\le[L:K]$. % 33]
\end{proposicion}

Recordemos que $\operatorname{Sub}(L/K)$ denota el conjunto de todas las subextensiones de $L/K$. \\
Si $G$ es un grupo, entonces vamos a denotar por $\operatorname{Sub}(G)$ al conjunto de todos los subgrupos de $G$ y si $H$ es un subgrupo de $G$, 
entonces $\operatorname{Sub}(G/H)$ es el conjunto de los subgrupos de $G$ que contienen a $H$. En realidad esta última notación es ambigua pues si $N$ es un subgrupo normal de $G$, entonces $\operatorname{Sub}(G/N)$ tiene dos significados: el conjunto de los subgrupos de $G$ que contienen a $N$ y el conjunto de los subgrupos del cociente $G/N$. 
El Teorema de la Correspondencia (Teorema 5.4 de GyA) nos muestra que esta ambigüedad no es muy grave.\\
Como $(\operatorname{Gal}(L/K))$ es un grupo, $\operatorname{Sub}(\operatorname{Gal}(L/K))$ es el conjunto de los subgrupos de $\operatorname{Gal}(L/K)$ y $\operatorname{Sub}(\operatorname{Gal}(L/K)/H)$ es el conjunto de los subgrupos de $\operatorname{Gal}(L/K)$ que contienen a $H$.

\begin{definicion}{Homomorfismo y anti-homomorfismo de conjuntos ordenados}
    Consideramos $\operatorname{Sub}(L/K)$ y $\operatorname{Sub}(G/H)$ como conjuntos ordenados por la inclusión. 
    Una aplicación $f:(A,\le)\rightarrow(B,\le)$ entre conjuntos ordenados se dice que es un \textbf{homomorfismo de conjuntos ordenados} si conserva el orden, es decir, 
    si para cada $x, y\in A$ tales que $x\le y$ se verifica que $f(x)\le f(y)$ y se dice que es un \textbf{anti-homomorfismo de conjuntos ordenados} si $f(x)\ge f(y)$ para todo $x, y\in A$ con $x\le y$. 
\end{definicion}

\begin{definicion}{Correspondencia de Galois}
    El par formado por las siguientes aplicaciones se llama \textbf{correspondencia de Galois de la extensión $L/K$}. 
    Si $L/K$ es una extensión de cuerpos entonces tenemos dos aplicaciones: 
    \begin{align*}
        (-)^{\circ} &:= \operatorname{Gal}(L/-) \colon \operatorname{Sub}(L/K) \longrightarrow \operatorname{Sub}(\operatorname{Gal}(L/K)) \\
        (-)^{\circ} &:= L^{(-)} \colon \operatorname{Sub}(\operatorname{Gal}(L/K)) \longrightarrow \operatorname{Sub}(L/K)
    \end{align*} 
    La aplicación que va para la derecha asocia $F\in \operatorname{Sub}(L/K)$ con 
    $$F^{\circ}=\operatorname{Gal}(L/F)=\{\sigma\in \operatorname{Gal}(L/K) \mid \sigma(x)=x \text{ para todo } x\in F\}$$ % 39, 40]
    y la que va para la izquierda asocia $H\in \operatorname{Sub}(\operatorname{Gal}(L/K))$ con
    $$H^{\circ}=L^{H}=\{a\in L \mid \sigma(a)=a \text{ para todo } \sigma\in H\}.$$ 
\end{definicion}


Veamos algunas propiedades de la correspondencia de Galois. % 43]

Recordemos que tanto la unidad de un anillo, como el neutro de un grupo o el subgrupo trivial del grupo lo denotamos siempre como $1$. 
En la siguiente proposición $1$ siempre denota el subgrupo trivial de $\operatorname{Gal}(L/K)$. 

\begin{proposicion}{Propiedades de la correspondencia de Galois}
    Sea $L/K$ una extensión de cuerpos y sea $G=\operatorname{Gal}(L/K)$. 
    La correspondencia de Galois $(-)^{\circ}:\operatorname{Sub}(L/K)\rightleftharpoons \operatorname{Sub}(G)$ satisface las siguientes propiedades, donde $X$ e $Y$ son ambos subextensiones de $L/K$ o ambos subgrupos de $G$: % 
    \begin{enumerate}
        \item $L^{\circ}=1$, $K^{\circ}=G$ y $1^{\circ}=L$.
        \item $(-)^{\circ}=\operatorname{Gal}(L/-)$ y $(-)^{\circ}=L^{(-)}$ son antihomomorfismos de conjuntos ordenados, es decir, si $X\subseteq Y$ entonces $Y^{\circ}\subseteq X^{\circ}$.
        \item $X\subseteq X^{\circ\circ}$ y $X^{\circ}=X^{\circ\circ\circ}$. 
        \item Las dos aplicaciones que forman la correspondencia de Galois se restringen a un anti-isomorfismo de conjuntos ordenados entre sus dos imágenes. 
    \end{enumerate}
\end{proposicion}

\begin{proof}
    Vamos a demostrar las propiedades paso a paso, recordando las definiciones explícitas de los operadores: si $E$ es un subcuerpo, $E^\circ = \operatorname{Gal}(L/E) = \{\sigma \in G \mid \sigma(x) = x \text{ para todo } x \in E\}$; y si $H$ es un subgrupo, $H^\circ = L^H = \{x \in L \mid \sigma(x) = x \text{ para todo } \sigma \in H\}$.

    \textbf{(1) Casos triviales:}
    \begin{itemize}
        \item $L^{\circ} = \operatorname{Gal}(L/L)$. El único automorfismo de $L$ que deja fijo todo elemento de $L$ es la identidad. Por tanto, $L^{\circ} = \{1\} = 1$.
        \item $K^{\circ} = \operatorname{Gal}(L/K)=G$. Por definición, este es el grupo de Galois total $G$.
        \item $1^{\circ} = L^{\{1\}}=\{\alpha\in L : 1(\alpha)=\alpha\} = L$. El conjunto de elementos de $L$ fijados por el automorfismo identidad es todo $L$.
    \end{itemize}

    \textbf{(2) Antihomomorfismos (Inversión del orden):}
    \begin{itemize}
        \item \textit{Para subcuerpos:} Supongamos que $E_1 \subseteq E_2$. Sea $\sigma \in E_2^{\circ} = \operatorname{Gal}(L/E_2)$. Por definición, $\sigma$ deja fijos todos los elementos de $E_2$. Como $E_1 \subseteq E_2$, $\sigma$ deja fijos todos los elementos de $E_1$. Esto implica que $\sigma \in \operatorname{Gal}(L/E_1) = E_1^{\circ}$. Por tanto, $E_2^{\circ} \subseteq E_1^{\circ}$.
        \item \textit{Para subgrupos:} Supongamos que $H_1 \subseteq H_2$. Sea $x \in H_2^{\circ} = L^{H_2}$. Por definición, $x$ es fijado por todo automorfismo de $H_2$. Como $H_1 \subseteq H_2$, $x$ es fijado por todo automorfismo de $H_1$. Luego $x \in L^{H_1} = H_1^{\circ}$. Por tanto, $H_2^{\circ} \subseteq H_1^{\circ}$.
    \end{itemize}

    \textbf{(3) Clausura y reflexividad:}
    Vamos a probar primero que $X \subseteq X^{\circ\circ}$.
    \begin{itemize}
        \item \textit{Si $X=E$ es un subcuerpo:} Sea $x \in E$. Para cualquier automorfismo $\sigma \in E^{\circ} = \operatorname{Gal}(L/E)$, se cumple por definición que $\sigma(x) = x$. Esto significa que $x$ está en el cuerpo fijo de $E^{\circ}$, es decir, $x \in (E^{\circ})^{\circ} = E^{\circ\circ}$. Por tanto, $E \subseteq E^{\circ\circ}$.
        \item \textit{Si $X=H$ es un subgrupo:} Sea $\sigma \in H$. Para cualquier elemento $x \in H^{\circ} = L^H$, se cumple por definición que $\sigma(x) = x$. Esto significa que $\sigma$ deja fijo todo el cuerpo $H^{\circ}$, luego $\sigma \in \operatorname{Gal}(L/H^{\circ}) = (H^{\circ})^{\circ} = H^{\circ\circ}$. Por tanto, $H \subseteq H^{\circ\circ}$.
    \end{itemize}
    Ahora probaremos que $X^{\circ} = X^{\circ\circ\circ}$.
    \begin{itemize}
        \item Sustituyendo $X$ por $X^{\circ}$ en la inclusión que acabamos de demostrar ($X \subseteq X^{\circ\circ}$), obtenemos inmediatamente que $X^{\circ} \subseteq (X^{\circ})^{\circ\circ} = X^{\circ\circ\circ}$.
        \item Por otro lado, tomemos la inclusión original $X \subseteq X^{\circ\circ}$. Si aplicamos a ambos lados el operador $(-)^{\circ}$, por la propiedad (2) de inversión del orden, la inclusión se da la vuelta: $(X^{\circ\circ})^{\circ} \subseteq X^{\circ}$, es decir, $X^{\circ\circ\circ} \subseteq X^{\circ}$.
        \item Al tener la doble inclusión, concluimos que $X^{\circ} = X^{\circ\circ\circ}$.
    \end{itemize}

    \textbf{(4) Anti-isomorfismo de las imágenes:}
    Llamemos \textbf{elementos cerrados} a aquellos subcuerpos o subgrupos que pertenecen a las imágenes de la correspondencia de Galois (es decir, aquellos de la forma $Y = X^{\circ}$). 
    
    Si tomamos un elemento cerrado $Y = X^{\circ}$ y le aplicamos dos veces el operador de Galois, obtenemos $Y^{\circ\circ} = (X^{\circ})^{\circ\circ} = X^{\circ\circ\circ}$. Por la propiedad (3), sabemos que $X^{\circ\circ\circ} = X^{\circ}$, con lo que deducimos que $Y^{\circ\circ} = Y$. 
    
    Esto demuestra que si restringimos las aplicaciones $(-)^{\circ}$ a sus respectivas imágenes, componerlas da la identidad (son biyecciones mutuamente inversas). Como además sabemos por (2) que invierten el orden, deducimos que establecen un anti-isomorfismo perfecto de conjuntos ordenados entre los subcuerpos cerrados y los subgrupos cerrados.
\end{proof}

\begin{definicion}{}
    Los elementos de las imágenes de las dos aplicaciones de la correspondencia de Galois se dice que son respectivamente 
    subextensiones cerradas en $L/K$ y subgrupos cerrados en $\operatorname{Gal}(L/K)$. % 9]
\end{definicion}

\begin{observacion}{El operador de clausura}
    Se tiene que un elemento $X$ (ya sea un cuerpo intermedio o un subgrupo) es cerrado si y solo si $X = X^{\circ\circ}$. 
    
    Llamamos a $X^{\circ\circ}$ la \textbf{clausura} de $X$ porque la doble aplicación de la correspondencia cumple las propiedades axiomáticas de un operador de clausura:
    \begin{itemize}
        \item \textbf{Extensivo:} $X \subseteq X^{\circ\circ}$. Para un cuerpo $M$, $M \subseteq L^{\text{Gal}(L/M)}$, lo que significa que el cuerpo fijo por los automorfismos que fijan $M$ es, al menos, tan grande como $M$.
        \item \textbf{Idempotente:} $(X^{\circ\circ})^{\circ\circ} = X^{\circ\circ}$. Una vez cerrado, el conjunto no crece más mediante aplicaciones sucesivas.
        \item \textbf{Monótono:} Si $A \subseteq B$, entonces $A^{\circ\circ} \subseteq B^{\circ\circ}$.
    \end{itemize}
\end{observacion}

\begin{corolario}{Anti-isomorfismo de conjuntos ordenados}
    Las aplicaciones de la correspondencia de Galois de una extensión de cuerpos $L/K$ se restringen a un \textbf{anti-isomorfismo} de conjuntos ordenados entre las subextensiones cerradas en $L/K$ y los subgrupos cerrados en $\text{Gal}(L/K)$.
\end{corolario}

\begin{observacion}{Cuerpos y grupos siempre cerrados (Obs. 5.1.10)}
    Por las propiedades de la correspondencia, los elementos extremos $L$, $1$ (el subgrupo identidad) y $\text{Gal}(L/K)$ son siempre cerrados en $L/K$.
    
    Sin embargo, el cuerpo base $K$ \textbf{no tiene por qué ser cerrado}. Por ejemplo, si $L \neq K$ pero $\text{Gal}(L/K) = 1$ (como ocurre en extensiones no normales o puramente inseparables), entonces la clausura de $K$ es:
    $$ K^{\circ\circ} = (K^{\circ})^{\circ} = 1^{\circ} = L \neq K $$
    En este caso, $K$ no es un elemento cerrado de la correspondencia.
\end{observacion}


\begin{proposicion}{Respecto a los grados de las extensiones y los índices de los subgrupos}
    Sea $L/K$ una extensión de cuerpos. 
    \begin{enumerate}
        \item Si $E_{1}\subseteq E_{2}$ son subextensiones de $L/K$ con $E_{2}/E_{1}$ finita entonces $[E_{1}^{\circ}:E_{2}^{\circ}]\le[E_{2}:E_{1}]$. % 1
        \item Si $H_{1}\le H_{2}$ son subgrupos de $\operatorname{Gal}(L/K)$ con $[H_{2}:H_{1}]<\infty$, entonces $[H_{1}^{\circ}:H_{2}^{\circ}]\le[H_{2}:H_{1}]$. % 18]
    \end{enumerate}
\end{proposicion}

\begin{proof}
    \textbf{1) Demostración de $[E_{1}^{\circ}:E_{2}^{\circ}]\le[E_{2}:E_{1}]$}

    Procedemos por inducción sobre el grado de la extensión $n = [E_{2} : E_{1}]$. 
    
    \textbf{Caso base ($n=1$):}
    Si $[E_{2} : E_{1}] = 1$, entonces $E_{2} = E_{1}$. Por definición, sus grupos de automorfismos asociados son idénticos: $E_{1}^{\circ} = E_{2}^{\circ}$. En consecuencia, el índice del subgrupo es $[E_{1}^{\circ} : E_{2}^{\circ}] = 1$, cumpliéndose la igualdad trivialmente.

    \textbf{Paso inductivo:}
    Supongamos que el resultado es válido para cualquier extensión de grado menor que $n$. Consideramos un elemento $\alpha \in E_{2} \setminus E_{1}$ y el cuerpo intermedio $E_{1}(\alpha)$. 

    Si $E_{1}(\alpha) \subsetneq E_{2}$, podemos descomponer la extensión en una torre: $E_{1} \subseteq E_{1}(\alpha) \subseteq E_{2}$. Por la propiedad multiplicativa de los grados:
    $$[E_{2} : E_{1}] = [E_{2} : E_{1}(\alpha)] \cdot [E_{1}(\alpha) : E_{1}]$$
    Como ambos factores son estrictamente menores que $n$, aplicamos la hipótesis de inducción a cada uno. Usando la propiedad de los índices de subgrupos en una torre:
    $$[E_{1}^{\circ} : E_{2}^{\circ}] = [E_{1}^{\circ} : E_{1}(\alpha)^{\circ}] \cdot [E_{1}(\alpha)^{\circ} : E_{2}^{\circ}] \le [E_{1}(\alpha) : E_{1}] \cdot [E_{2} : E_{1}(\alpha)] = [E_{2} : E_{1}]$$

    \textbf{Caso de la extensión simple ($E_{2} = E_{1}(\alpha)$):}
    Sea $p = \operatorname{Min}_{E_{1}}(\alpha)$ el polinomio mínimo de $\alpha$ sobre $E_{1}$, con grado $s = \operatorname{gr}(p) = [E_{2} : E_{1}]$. Definimos $R$ como el conjunto de raíces de $p$ contenidas en $L$.
    
    Consideramos la aplicación entre el conjunto de clases laterales y las raíces:
    $$ \psi : E_{1}^{\circ} / E_{2}^{\circ} \longrightarrow R, \quad \psi(\sigma E_{2}^{\circ}) = \sigma(\alpha) $$
    
    \begin{itemize}
        \item \textbf{Buena definición:} Supongamos que $\sigma E_{2}^{\circ} = \tau E_{2}^{\circ}$. Esto implica que $\tau^{-1}\sigma \in E_{2}^{\circ}$. Por definición, el grupo $E_{2}^{\circ} = \text{Gal}(L/E_{2})$ fija todos los elementos de $E_{2}$. Como $\alpha \in E_{2}$, se tiene que $\tau^{-1}\sigma(\alpha) = \alpha$, lo que implica $\sigma(\alpha) = \tau(\alpha)$. Además, como $\sigma \in E_{1}^{\circ}$ fija $E_{1}$ y $\alpha$ es raíz de $p \in E_{1}[X]$, $\sigma(\alpha)$ es obligatoriamente otra raíz de $p$ en $L$, por lo que la imagen está en $R$.
        
        \item \textbf{Inyectividad:} Supongamos que $\psi(\sigma E_{2}^{\circ}) = \psi(\tau E_{2}^{\circ})$, es decir, $\sigma(\alpha) = \tau(\alpha)$. Multiplicando por la izquierda por el inverso, tenemos $\tau^{-1}\sigma(\alpha) = \alpha$. Dado que $\tau^{-1}\sigma$ ya fijaba $E_{1}$ (por ser composición de elementos de $E_{1}^{\circ}$), ahora también fija al generador $\alpha$. Por tanto, fija a todo el cuerpo $E_{1}(\alpha) = E_{2}$, lo que significa que $\tau^{-1}\sigma \in E_{2}^{\circ}$. Esto demuestra que $\sigma E_{2}^{\circ} = \tau E_{2}^{\circ}$.
    \end{itemize}

    Al ser $\psi$ una aplicación inyectiva, el cardinal del dominio no puede superar al del codominio:
    $$ [E_{1}^{\circ} : E_{2}^{\circ}] \le |R| \le \operatorname{gr}(p) = [E_{2} : E_{1}] $$
    donde la última desigualdad se debe a que un polinomio de grado $s$ tiene a lo sumo $s$ raíces.
    
    \vspace{0.3 mm}


    \textbf{2) Demostración de $[H_{1}^{\circ}:H_{2}^{\circ}]\le[H_{2}:H_{1}]$}\\
    \textbf{Aclaración previa sobre los representantes $\tau_i$:}
    Por hipótesis, el índice del subgrupo es finito y vale $[H_{2}:H_{1}] = n$. En la teoría de grupos, el índice nos dice exactamente en cuántos "trozos" (clases laterales) podemos dividir el grupo grande ($H_2$) usando el subgrupo pequeño ($H_1$). 
    Por tanto, el grupo $H_2$ se particiona en $n$ clases laterales por la izquierda. Los elementos $\tau_1, \tau_2, \dots, \tau_n \in H_2$ son los \textbf{representantes} elegidos para cada una de esas clases, de modo que $H_{2}/H_{1}=\{\tau_{1}H_{1},\dots,\tau_{n}H_{1}\}$. 
    Como el propio subgrupo $H_1$ es una de estas clases laterales, resulta muy conveniente elegir el elemento neutro (la identidad de Galois) como su representante, por lo que fijamos $\tau_{1}=1$.

    \vspace{0.4cm}
    \textbf{Desarrollo formal por reducción al absurdo:}
    
    Nuestro objetivo es demostrar que $[H_{1}^{\circ}:H_{2}^{\circ}] \le n$. Supongamos lo contrario, es decir, que $[H_{1}^{\circ}:H_{2}^{\circ}] > n$. 
    Recordemos que $H_1^\circ$ y $H_2^\circ$ son subcuerpos de $L$ denominados \textbf{cuerpos fijos} por la acción de los subgrupos $H_1, H_2 \le \operatorname{Gal}(L/K)$. 

Formalmente, se definen como los conjuntos de elementos de $L$ que permanecen invariantes ante la acción de cada uno de los automorfismos del subgrupo correspondiente:

\begin{equation*}
    H_1^\circ = \{ a \in L \mid \sigma(a) = a, \quad \forall \sigma \in H_1 \}
\end{equation*}
\begin{equation*}
    H_2^\circ = \{ a \in L \mid \sigma(a) = a, \quad \forall \sigma \in H_2 \}
\end{equation*}

Donde cada $\sigma \in \operatorname{Gal}(L/K)$ es un $K$-automorfismo del cuerpo $L$. Por las propiedades de la correspondencia de Galois, estos conjuntos siempre tienen estructura de cuerpo.

    Si la dimensión del espacio vectorial $H_{1}^{\circ}$ sobre el cuerpo base $H_{2}^{\circ}$ es estrictamente mayor que $n$, significa que podemos encontrar al menos $n+1$ 
    elementos en $H_{1}^{\circ}$ que sean linealmente independientes sobre $H_{2}^{\circ}$ (aquí $H_{2}^{\circ}$ funciona como el cuerpo donde cogemos los escalares). 
    Llamemos a estos elementos $\alpha_{1},\dots,\alpha_{n},\alpha_{n+1} \in H_{1}^{\circ}$.

    Evaluemos todos nuestros representantes $\tau_i$ en cada uno de estos elementos y formemos la siguiente matriz $A$ de tamaño $n \times (n+1)$ con coeficientes en $L$:
    $$A=\begin{pmatrix}
    \tau_{1}(\alpha_{1}) & \tau_{1}(\alpha_{2}) & \dots & \tau_{1}(\alpha_{n+1})\\ 
    \tau_{2}(\alpha_{1}) & \tau_{2}(\alpha_{2}) & \dots & \tau_{2}(\alpha_{n+1})\\ 
    \vdots & \vdots & \ddots & \vdots \\ 
    \tau_{n}(\alpha_{1}) & \tau_{n}(\alpha_{2}) & \dots & \tau_{n}(\alpha_{n+1})
    \end{pmatrix}$$
    
    Cualquier sistema de ecuaciones lineales homogéneo con más incógnitas (columnas = $n+1$) que ecuaciones (filas = $n$) tiene siempre soluciones no triviales. Por tanto, el núcleo de esta matriz contiene vectores distintos de cero.
    
    \textbf{El argumento de minimidad (Truco de Artin):}
    Sea $r$ el rango por columnas de $A$. Evidentemente $r \le n < n+1$. 
    Reordenando los elementos $\alpha_{i}$ si fuera necesario, podemos suponer sin pérdida de generalidad que las primeras $r$ columnas de $A$ son linealmente independientes y que la columna $r+1$ es una combinación lineal exacta de estas $r$ primeras columnas. 
    
    Esto nos garantiza que existe un vector solución $a \in L^{n+1}$ no nulo con una estructura muy específica:
    $$ a=(a_{1},\dots,a_{r},1,0,\dots,0)^T $$
    donde el 1 está en la posición $r+1$. Se cumple que $Aa=0$.
    
    \textbf{Análisis de los coeficientes de la solución:}
    Fijémonos en la primera fila del sistema $Aa=0$. Al ser $\tau_{1}=1$ (la identidad), la primera fila de la matriz es simplemente $(\alpha_1, \alpha_2, \dots, \alpha_{n+1})$. Multiplicando por el vector $a$, obtenemos:
    $$\alpha_{1}a_{1}+\dots+\alpha_{r}a_{r}+\alpha_{r+1} \cdot 1 = 0$$
    
    Si todos los coeficientes $a_1, \dots, a_r$ pertenecieran al cuerpo base $H_{2}^{\circ}$, tendríamos una combinación lineal nula de elementos $\alpha_i$ sobre el cuerpo $H_{2}^{\circ}$. Pero habíamos elegido los $\alpha_i$ precisamente para que fueran linealmente independientes sobre $H_{2}^{\circ}$, lo cual haría esto imposible. 
    Concluimos que al menos uno de los coeficientes no pertenece a $H_{2}^{\circ}$. Reordenando de nuevo si hace falta, supongamos que $a_{1}\notin H_{2}^{\circ}$.
    
    ¿Qué significa que $a_{1}\notin H_{2}^{\circ}$? Por la definición de cuerpo fijo, significa que existe algún automorfismo $\sigma \in H_{2}$ que "mueve" a $a_1$, es decir, $\sigma(a_{1})\ne a_{1}$.

    \textbf{La acción del automorfismo $\sigma$ sobre el sistema:}
    Tomemos la ecuación vectorial completa $Aa=0$ y apliquémosle el automorfismo $\sigma$ a todo el sistema. Dado que $\sigma(0)=0$, obtenemos $\sigma(A)\sigma(a) = 0$.
    
    Vamos a ver qué forma tiene la matriz transformada $\sigma(A)$. Su elemento genérico en la fila $i$ y columna $j$ es $\sigma(\tau_i(\alpha_j))$. 
    Notemos que, al ser $\sigma \in H_2$, el conjunto de clases laterales por la izquierda $\{\sigma\tau_1 H_1, \dots, \sigma\tau_n H_1\}$ es exactamente el mismo conjunto original de clases laterales (solo se han permutado). Por tanto, para cada índice $i$, existe un único índice $k = \rho(i)$ y un elemento $h_i \in H_1$ tales que:
    $$ \sigma\tau_i = \tau_{\rho(i)} h_i $$
    
    Ahora evaluamos esto en $\alpha_j$. Recordemos que $\alpha_j \in H_{1}^{\circ}$, lo que significa que es invariante (se queda fijo) ante cualquier elemento de $H_1$, incluyendo $h_i$. Por tanto $h_i(\alpha_j) = \alpha_j$. Sustituyendo:
    $$ \sigma(\tau_i(\alpha_j)) = \tau_{\rho(i)}(h_i(\alpha_j)) = \tau_{\rho(i)}(\alpha_j) $$
    
    Esto demuestra un hecho crucial: la matriz $\sigma(A)$ es exactamente la misma matriz $A$, pero con las \textbf{filas permutadas} según la permutación $\rho$. Un sistema de ecuaciones no cambia sus soluciones por cambiar el orden de las filas, lo que implica directamente que:
    $$ A\sigma(a)=0 $$

    \textbf{La contradicción final:}
    Tenemos ahora dos soluciones distintas para el sistema matricial original:
    \begin{itemize}
        \item $a = (a_1, \dots, a_r, 1, 0, \dots, 0)^T$
        \item $\sigma(a) = (\sigma(a_1), \dots, \sigma(a_r), \sigma(1), 0, \dots, 0)^T = (\sigma(a_1), \dots, \sigma(a_r), 1, 0, \dots, 0)^T$
    \end{itemize}
    (Dado que $\sigma$ es un automorfismo de cuerpos, $\sigma(1)=1$ y $\sigma(0)=0$).
    
    Como el sistema es lineal, la resta de dos soluciones es otra solución. Restamos ambos vectores:
    $$ a-\sigma(a) = \begin{pmatrix}
    a_{1}-\sigma(a_{1}) \\ 
    \vdots \\ 
    a_{r}-\sigma(a_{r}) \\ 
    1-1 \\ 
    0 \\ 
    \vdots \\ 
    0
    \end{pmatrix} = \begin{pmatrix}
    a_{1}-\sigma(a_{1}) \\ 
    \vdots \\ 
    a_{r}-\sigma(a_{r}) \\ 
    0 \\ 
    0 \\ 
    \vdots \\ 
    0
    \end{pmatrix} $$
    
    Observemos este nuevo vector solución. Es distinto del vector nulo porque habíamos garantizado que $\sigma(a_{1}) \neq a_{1}$, luego su primera componente $a_{1}-\sigma(a_{1}) \ne 0$.
    Sin embargo, tiene un cero en la posición $r+1$. Esto significa que acabamos de encontrar una combinación lineal nula utilizando \textbf{únicamente} las primeras $r$ columnas de la matriz $A$ (sin necesitar la columna $r+1$).
    
    Pero esto es una contradicción flagrante, porque habíamos elegido $r$ precisamente de forma que las primeras $r$ columnas de $A$ fueran \textbf{linealmente independientes}.
    
    Esta contradicción proviene de nuestra suposición inicial de que $[H_{1}^{\circ}:H_{2}^{\circ}] > n$. Por lo tanto, dicha suposición debe ser falsa, quedando demostrado que $[H_{1}^{\circ}:H_{2}^{\circ}] \le n = [H_{2}:H_{1}]$.
\end{proof}

\begin{corolario}{}
    Sea $L/K$ una extensión de cuerpos. % 46]
    \begin{enumerate}
        \item Si $K\subseteq E_{1}\subseteq E_{2}\subseteq L$ es una torre de cuerpos, con $[E_{2}:E_{1}]<\infty$ y $E_{1}$ cerrado en $L/K$ entonces $E_{2}$ es cerrado en $L/K$ y $[E_{1}^{\circ}:E_{2}^{\circ}]=[E_{2}:E_{1}]$. % 4
        \item Si $H_{1}\le H_{2}\le \operatorname{Gal}(L/K)$ son subgrupos de $\operatorname{Gal}(L/K)$ con $[H_{2}:H_{1}]<\infty$ y $H_{1}$ cerrado en $L/K$ entonces $H_{2}$ es cerrado en $L/K$ y $[H_{1}^{\circ}:H_{2}^{\circ}]=[H_{2}:H_{1}]$. % 48]
    \end{enumerate}
\end{corolario}

\begin{proof}
    \textbf{(1) Demostración para subextensiones:}

    Queremos probar que si $E_{1}$ es cerrado ($E_{1} = E_{1}^{\circ\circ}$) y la extensión es finita, entonces $E_{2}$ también es cerrado y se da la igualdad de grados. 

    \begin{itemize}
        \item \textbf{Paso 1: Aplicación de la Proposición 5.5 (parte 1).} 
        Consideramos la inclusión de subextensiones $E_{1} \subseteq E_{2}$. Por la proposición demostrada anteriormente, sabemos que el índice del subgrupo asociado no supera al grado de la extensión:
        \begin{equation}
            [E_{1}^{\circ} : E_{2}^{\circ}] \le [E_{2} : E_{1}] \tag{I}
        \end{equation}

        \item \textbf{Paso 2: Aplicación de la Proposición 5.5 (parte 2).}
        Ahora aplicamos la segunda parte de la proposición a los subgrupos $H_{1} = E_{2}^{\circ}$ y $H_{2} = E_{1}^{\circ}$. Notemos que al ser $E_1 \subseteq E_2$, se invierte el orden en los grupos: $E_{2}^{\circ} \subseteq E_{1}^{\circ}$. La proposición nos dice que el grado de la extensión de los cuerpos fijos no supera al índice de los grupos:
        \begin{equation}
            [E_{2}^{\circ\circ} : E_{1}^{\circ\circ}] \le [E_{1}^{\circ} : E_{2}^{\circ}] \tag{II}
        \end{equation}

        \item \textbf{Paso 3: Uso de la hipótesis de clausura.}
        Como $E_{1}$ es cerrado en $L/K$ por hipótesis, se cumple que $E_{1} = E_{1}^{\circ\circ}$. Sustituyendo esto en la desigualdad (II) y encadenándola con la (I), obtenemos:
        \begin{equation}
            [E_{2}^{\circ\circ} : E_{1}] \le [E_{1}^{\circ} : E_{2}^{\circ}] \le [E_{2} : E_{1}] \tag{III}
        \end{equation}

        \item \textbf{Paso 4: Conclusión por inclusión y dimensiones.}
        Sabemos que para cualquier subextensión se cumple la inclusión $E_{2} \subseteq E_{2}^{\circ\circ}$. En una torre de cuerpos $E_1 \subseteq E_2 \subseteq E_2^{\circ\circ}$, si el grado total $[E_2^{\circ\circ} : E_1]$ es menor o igual al grado intermedio $[E_2 : E_1]$, la única posibilidad lógica es que los cuerpos sean idénticos:
        $$ E_{2} = E_{2}^{\circ\circ} $$
        Esto demuestra que $E_{2}$ es \textbf{cerrado}. Al ser idénticos, sus grados son iguales, lo que fuerza a que todas las desigualdades en (III) se conviertan en igualdades, probando que $[E_{1}^{\circ} : E_{2}^{\circ}] = [E_{2} : E_{1}]$.
    \end{itemize}

    \vspace{0.4cm}
    \textbf{(2) Demostración para subgrupos:}

    El razonamiento es totalmente simétrico al anterior, intercambiando los roles de cuerpos y grupos.

    \begin{itemize}
        \item Por la Proposición 5.5 (parte 2) aplicada a $H_{1} \le H_{2}$, tenemos que $[H_{1}^{\circ} : H_{2}^{\circ}] \le [H_{2} : H_{1}]$.
        \item Aplicando la parte 1 de la misma proposición a los cuerpos $E_1 = H_2^\circ$ y $E_2 = H_1^\circ$ (recordando que $H_2^\circ \subseteq H_1^\circ$), obtenemos $[H_{2}^{\circ\circ} : H_{1}^{\circ\circ}] \le [H_{1}^{\circ} : H_{2}^{\circ}]$.
        \item Como $H_{1}$ es cerrado, $H_{1} = H_{1}^{\circ\circ}$. Combinando las desigualdades:
        $$ [H_{2}^{\circ\circ} : H_{1}] \le [H_{1}^{\circ} : H_{2}^{\circ}] \le [H_{2} : H_{1}] $$
        \item Dada la inclusión natural $H_{2} \subseteq H_{2}^{\circ\circ}$, y que el índice de la clausura no puede superar al del grupo original respecto al mismo subgrupo cerrado $H_1$, concluimos que $H_{2} = H_{2}^{\circ\circ}$.
        \item Por tanto, $H_{2}$ es \textbf{cerrado} y se verifica la igualdad de índices $[H_{1}^{\circ} : H_{2}^{\circ}] = [H_{2} : H_{1}]$.
    \end{itemize}
\end{proof}

Como consecuencia del segundo apartado del Corolario 5.6 y el primero de la Proposición 5.3 se tiene el siguiente corolario: % 52]

\begin{corolario}{Cierre de subgrupos finitos}
    Todo subgrupo finito de $\operatorname{Gal}(L/K)$ es cerrado en $L/K$. %
\end{corolario}

\begin{proof}
    Sea $H$ un subgrupo finito de $\operatorname{Gal}(L/K)$. Para demostrar que es cerrado, aplicaremos los resultados de cierre obtenidos anteriormente:

    \begin{itemize}
        \item Consideramos el subgrupo trivial $H_1 = \{1\}$. 
        \item Sabemos que el subgrupo trivial $\{1\}$ es siempre un subgrupo \textbf{cerrado} en $\operatorname{Gal}(L/K)$. Esto se debe a que su cuerpo fijo es todo el cuerpo superior, $\{1\}^\circ = L$, y el grupo de automorfismos que fijan $L$ es únicamente la identidad, $L^\circ = \{1\}$.
        \item Por hipótesis, $H$ es un subgrupo finito, lo que implica que el índice del subgrupo trivial dentro de $H$ es finito: $[H : \{1\}] = |H| < \infty$.
        \item Aplicamos ahora el segundo apartado del corolario anterior, el cual establece que si $H_1 \le H$ es una inclusión de subgrupos donde el subgrupo menor ($H_1$) es cerrado y el índice $[H : H_1]$ es finito, entonces el subgrupo mayor ($H$) es obligatoriamente cerrado.
    \end{itemize}

    Por lo tanto, al ser $\{1\}$ cerrado y $[H:\{1\}]$ finito, concluimos que $H$ es cerrado en $L/K$.
\end{proof}


%%%% SECCIÓN 5.2 %%%%% EXTENSIONES DE GALOIS %%%%

\section{Extensiones de Galois} 

\begin{definicion}{Extensión de Galois}
    Una extensión de Galois es una extensión de cuerpos que es normal y separable. 
\end{definicion}

\begin{observacion}{} 
    Obsérvese que toda extensión de Galois es algebraica (por ser separable). 
\end{observacion}

La siguiente proposición es consecuencia inmediata de que la clase de extensiones normales 
es cerrada para levantamientos y de que la clase de extensiones separables también lo es.

\begin{proposicion}{} % 
    La clase de extensiones de Galois es cerrada para levantamientos. % 
\end{proposicion}

El siguiente teorema caracteriza las extensiones de Galois. % 8]

\begin{teorema}{Condiciones equivalentes para una extensión de Galois} 
    Las siguientes condiciones son equivalentes para una extensión de cuerpos $L/K$ con $G=\operatorname{Gal}(L/K)$: % 10]
    \begin{enumerate}
        \item $L/K$ es una extensión de Galois. % 11]
        \item $L/E$ es una extensión de Galois para todo $E\in \operatorname{Sub}(L/K)$. 
        \item $L/K$ es algebraica y toda subextensión de $L/K$ es cerrada. 
        \item $L/K$ es algebraica y $K$ es una subextensión cerrada de $L/K$. 
        \item $L/K$ es algebraica y $G^{\circ}=K$, o sea, si $\alpha\in L$ satisface $\sigma(\alpha)=\alpha$ para todo $\sigma\in G$ entonces $\alpha\in K$. % 15]
        \item $L/K$ es algebraica y para todo $\alpha\in L\setminus K$ existe $\sigma\in \operatorname{Gal}(L/K)$ tal que $\sigma(\alpha)\ne\alpha$. % 16]
    \end{enumerate}
\end{teorema}

\begin{proof}
    Vamos a demostrar la equivalencia formando un ciclo de implicaciones lógicas y demostrando las equivalencias directas cuando sea conveniente.

    \vspace{0.3cm}
    \textbf{(1) $\implies$ (2): Toda subextensión es de Galois.} \\
    Recordemos la definición de una clase de extensiones cerrada para levantamientos: si una extensión $L_1/K$ pertenece a una clase $\mathcal{C}$, y tenemos otra extensión admisible $L_2/K$, entonces la extensión "levantada" hacia el cuerpo compuesto, $L_1 L_2 / L_2$, también pertenece a la clase $\mathcal{C}$.

    Sabemos que la clase de las extensiones de Galois es cerrada para levantamientos (al igual que lo son las extensiones finitas, algebraicas, finitamente generadas y simples). 

    Para nuestro caso, definimos nuestras extensiones admisibles sobre el cuerpo base $K$:
    \begin{itemize}
        \item Sea $L_1 = L$. Por la hipótesis (1), sabemos que $L/K$ es una extensión de Galois, luego $L_1/K \in \mathcal{C}$.
        \item Sea $L_2 = E$, donde $E$ es un cuerpo intermedio ($K \subseteq E \subseteq L$). Esta es la extensión hacia la que queremos levantar.
    \end{itemize}

    Calculamos el cuerpo compuesto $L_1 L_2$:
    $$L_1 L_2 = L \cdot E$$
    Como $E$ es un subcuerpo de $L$ (es decir, $E \subseteq L$), el menor cuerpo que contiene a ambos es simplemente el cuerpo más grande.
     Por tanto, el cuerpo compuesto es $L_1 L_2 = L$.

    Aplicando la definición de la propiedad de cierre bajo levantamientos:
    $$L_1/K \text{ es de Galois} \implies L_1 L_2 / L_2 \text{ es de Galois}$$
    
    Sustituyendo nuestros cuerpos concretos en la implicación, obtenemos:
    $$L/K \text{ es de Galois} \implies L/E \text{ es de Galois}$$

    Lo cual demuestra de forma directa y estructural que, para todo cuerpo intermedio $E \in \operatorname{Sub}(L/K)$, la subextensión $L/E$ es de Galois.

    \textbf{(2) $\implies$ (3): Toda subextensión es cerrada.} \\
    Supongamos que $L/E$ es una extensión de Galois para cualquier cuerpo intermedio $E$. Queremos demostrar que $E$ es una subextensión cerrada, es decir, que $E = E^{\circ\circ}$.
    
    Por las propiedades de la correspondencia de Galois, siempre se tiene la inclusión trivial $E \subseteq E^{\circ\circ}$. 
    Para demostrar la igualdad, probaremos la inclusión contraria viendo que si un elemento $\alpha$ no está en $E$, tampoco puede estar en $E^{\circ\circ}$. 
    Recordemos que $E^{\circ\circ}$ es el cuerpo fijo del grupo $\operatorname{Gal}(L/E)$. Por tanto, demostrar que $\alpha \notin E^{\circ\circ}$ equivale a encontrar al menos un automorfismo $\sigma \in \operatorname{Gal}(L/E)$ que "mueva" a $\alpha$ (es decir, $\sigma(\alpha) \neq \alpha$).

    Sea $\alpha \in L \setminus E$ y sea $p = \operatorname{Min}_E(\alpha)$.
    \begin{itemize}
        \item Como $L/E$ es normal por hipótesis, $p$ factoriza completamente en $L$, es decir, todas sus raíces están en $L$.
        \item Como $L/E$ es separable y $\alpha \notin E$ (luego el grado de $p$ es estrictamente mayor que 1), $p$ no tiene raíces múltiples.
    \end{itemize}
    Por consiguiente, existe obligatoriamente otra raíz $\beta \in L$ del mismo polinomio $p$, con $\beta \neq \alpha$.
    
    Por la Proposición 1.10 (isomorfismo de raíces conjugadas), sabemos que existe un $E$-isomorfismo $\tau: E(\alpha) \xrightarrow{\simeq} E(\beta)$ tal que $\tau(\alpha) = \beta$.
    Sea $\overline{L}$ la clausura algebraica de $L$. Como la extensión $L/E(\alpha)$ es algebraica, el Teorema de Extensión de Homomorfismos garantiza que podemos extender $\tau$ a un homomorfismo global $\sigma: L \rightarrow \overline{L}$.
    
    Pero, al ser $L/E$ una extensión normal, cualquier $E$-homomorfismo de $L$ en su clausura algebraica cumple que $\sigma(L) = L$. Esto convierte a $\sigma$ en un $E$-automorfismo de $L$, es decir, $\sigma \in \operatorname{Gal}(L/E)$.
    Evaluando en nuestro elemento: $\sigma(\alpha) = \beta \neq \alpha$. 
    Hemos encontrado el automorfismo que mueve a $\alpha$, luego $\alpha \notin E^{\circ\circ}$. Esto prueba que $E^{\circ\circ} \subseteq E$ y, por tanto, $E = E^{\circ\circ}$.

    \vspace{0.3cm}
    \textbf{(3) $\implies$ (4): El cuerpo base es cerrado.} \\
    Si toda subextensión de $L/K$ es cerrada, en particular el propio cuerpo base $K$ es una subextensión (la más pequeña posible). Por lo tanto, $K$ es cerrado. (La extensión es algebraica por hipótesis heredada).

    \vspace{0.3cm}
    \textbf{(4) $\implies$ (5): Igualdad del cuerpo fijo.} \\
    Asumimos que $K$ es cerrado, lo que por definición significa que $K = K^{\circ\circ}$.
    Desplegando la notación de la correspondencia de Galois:
    \begin{itemize}
        \item $K^\circ$ es el grupo de automorfismos que fijan $K$. Esto es, por definición, todo el grupo de Galois: $K^\circ = \operatorname{Gal}(L/K) = G$.
        \item Aplicando el segundo círculo, $K^{\circ\circ} = (K^\circ)^\circ = G^\circ$.
    \end{itemize}
    Sustituyendo en la hipótesis, obtenemos directamente $K = G^\circ$. Esto significa exactamente que el conjunto de elementos de $L$ que quedan fijos por todos los elementos de $G$ es exactamente $K$.

    \vspace{0.3cm}
    \textbf{(5) $\iff$ (6): Equivalencia lógica del cuerpo fijo.} \\
    Esta equivalencia es un mero parafraseo de la definición de cuerpo fijo.
    \begin{itemize}
        \item $(5) \implies (6)$: Si $G^\circ = K$, significa que si un elemento está en $L$ pero no en $K$ ($\alpha \in L \setminus K$), entonces no puede pertenecer al cuerpo fijo $G^\circ$. Al no estar en el cuerpo fijo, debe existir al menos un automorfismo en el grupo $G$ que no lo deje fijo, es decir, $\exists \sigma \in G$ tal que $\sigma(\alpha) \neq \alpha$.
        \item $(6) \implies (5)$: Si para todo $\alpha \notin K$ existe un automorfismo que lo mueve, entonces ningún elemento fuera de $K$ puede pertenecer al cuerpo fijo $G^\circ$. Por otro lado, todos los elementos de $K$ quedan fijos por definición de los $K$-automorfismos de $G$. Por tanto, el cuerpo fijo es exactamente $K$ ($G^\circ = K$).
    \end{itemize}

    \vspace{0.3cm}
    \textbf{(5) $\implies$ (1): Caracterización de Galois.} \\
    Esta es la implicación fundamental. Supongamos que $G^\circ = K$. Queremos demostrar que $L/K$ es de Galois, es decir, que es simultáneamente normal y separable. 
    Para ello, tomamos un elemento arbitrario $\alpha \in L$, sea $p = \operatorname{Min}_K(\alpha)$ su polinomio mínimo, con $n = \operatorname{gr}(p)$. Debemos demostrar que $p$ factoriza completamente en $L$ (normalidad) y que todas sus raíces son simples (separabilidad).
    
    Definimos $R = \{\alpha_1, \alpha_2, \dots, \alpha_r\}$ como el conjunto de las \textbf{distintas} raíces de $p$ que residen dentro de $L$. (Sabemos que $r \ge 1$ porque al menos el propio $\alpha$ está en $R$).
    
    Construimos el polinomio $q(X)$ usando exclusivamente estas raíces distintas:
    $$ q(X) = (X-\alpha_1)(X-\alpha_2)\cdots(X-\alpha_r) $$
    Por construcción, $q(X)$ es un polinomio de $L[X]$ que no tiene raíces múltiples y cuyo grado es $r$.
    
    Consideremos la acción de cualquier automorfismo $\sigma \in G = \operatorname{Gal}(L/K)$ sobre el polinomio $q(X)$. Como $p \in K[X]$, sus coeficientes están fijos por $\sigma$, por lo que $\sigma(p) = p$. Esto implica que $\sigma$ envía raíces de $p$ en raíces de $p$. Dado que $\sigma(L) = L$, $\sigma$ induce una permutación sobre el conjunto finito $R$.
    
    Si aplicamos $\sigma$ a los coeficientes de $q(X)$, lo que estamos haciendo es aplicar $\sigma$ a sus raíces:
    $$ \sigma(q(X)) = (X-\sigma(\alpha_1))(X-\sigma(\alpha_2))\cdots(X-\sigma(\alpha_r)) $$
    Como $\sigma$ simplemente permuta los elementos de $R$, los factores son exactamente los mismos pero en distinto orden. Por tanto, $\sigma(q(X)) = q(X)$.
    
    Esto significa que todos y cada uno de los coeficientes del polinomio $q(X)$ quedan fijos por \textbf{todos} los automorfismos $\sigma \in G$.
    Por la hipótesis (5), el conjunto de elementos de $L$ que quedan fijos por todo el grupo de Galois es exactamente el cuerpo base $K$. Por consiguiente, los coeficientes de $q(X)$ pertenecen a $K$, lo que implica que $q \in K[X]$.
    
    Recapitulemos: tenemos un polinomio $q \in K[X]$ que tiene a $\alpha$ como raíz (pues $\alpha \in R$). Por la propia definición de polinomio mínimo, el mínimo $p(X)$ debe dividir a cualquier otro polinomio en $K[X]$ que anule a $\alpha$. Por tanto, $p$ divide a $q$.
    
    Sin embargo, analizamos los grados:
    $$ \operatorname{gr}(q) = r \quad \text{y} \quad \operatorname{gr}(p) = n $$
    Como $R$ contiene, a lo sumo, a todas las raíces de $p$, es obvio que $r \le n$. 
    El único modo en que un polinomio $p$ de grado $n$ pueda dividir a un polinomio $q$ de grado $r \le n$ (siendo ambos mónicos) es que sean el mismo polinomio.
    
    Concluimos que $p(X) = q(X)$ y $r = n$.
    Esta igualdad nos da automáticamente las dos propiedades que buscábamos:
    \begin{itemize}
        \item \textit{Normalidad:} Como $p=q$ y $q$ se construyó multiplicando factores $(X-\alpha_i)$ con $\alpha_i \in L$, $p$ descompone completamente en $L$.
        \item \textit{Separabilidad:} Como $q$ se construyó tomando raíces distintas (sin repeticiones), $p$ no tiene raíces múltiples.
    \end{itemize}
    Dado que esto es cierto para todo $\alpha \in L$, la extensión $L/K$ es de Galois.
\end{proof}

La siguiente proposición muestra criterios para decidir si una extensión es de Galois para el caso de extensiones finitas. % 29]

\begin{proposicion}{Criterios para una extensión de Galois finita} 
    Las siguientes condiciones son equivalentes para una extensión finita $L/K$: 
    \begin{enumerate}
        \item $L/K$ es una extensión de Galois. 
        \item $[L:K]=|\operatorname{Gal}(L/K)|$. 
        \item $[L:E]=|\operatorname{Gal}(L/E)|$ para todo $E\in \operatorname{Sub}(L/K)$. 
    \end{enumerate}
\end{proposicion}

\begin{proof}
    Vamos a demostrar en primer lugar la equivalencia entre (1) y (2) analizando la cadena de desigualdades fundamentales que relaciona el orden del grupo de Galois, el grado de separabilidad y el grado de la extensión.

    \vspace{0.3cm}
    \textbf{Demostración de (1) $\iff$ (2):}

    Sea $\overline{L}$ una clausura algebraica de $L$. 
    Recordemos que el grupo de Galois, $\operatorname{Gal}(L/K)$, está formado por todos los $K$-automorfismos de $L$. Todo automorfismo $\tau \in \operatorname{Gal}(L/K)$ puede verse de forma natural como un $K$-homomorfismo (o inmersión) del cuerpo $L$ en su clausura algebraica $\overline{L}$, con la particularidad de que su imagen es exactamente $L$ ($\tau(L) = L$).

    Sea $S_{K}^{L}$ el conjunto de todos los $K$-homomorfismos de $L$ en $\overline{L}$. (En la notación original, si $\sigma$ es la inclusión de $K$ en $\overline{L}$, este conjunto se denota como $S_{\sigma}^{L}$).
    Como todo elemento de $\operatorname{Gal}(L/K)$ es una de estas inmersiones, tenemos la inclusión de conjuntos:
    $$ \operatorname{Gal}(L/K) \subseteq S_{K}^{L} $$
    Lo que implica que el número de automorfismos es menor o igual al número total de inmersiones:
    $$ |\operatorname{Gal}(L/K)| \le |S_{K}^{L}| $$

    Por teoría de cuerpos (Proposición 5.2), el número total de $K$-homomorfismos de $L$ en $\overline{L}$ se define como el \textbf{grado de separabilidad} de la extensión, denotado como $[L:K]_s$. 
    A su vez, es un resultado conocido que el grado de separabilidad nunca supera al grado total de la extensión $[L:K]$.
    
    Juntando todo esto, obtenemos la cadena de desigualdades fundamental:
    \begin{equation}
        |\operatorname{Gal}(L/K)| \le [L:K]_s \le [L:K] \tag{*}
    \end{equation}

    Analicemos cuándo estas dos desigualdades se convierten en igualdades estrictas:
    \begin{itemize}
        \item \textbf{La segunda desigualdad ($[L:K]_s \le [L:K]$):} 
        El Teorema 4.12 nos garantiza que el número de inmersiones coincide con el grado de la extensión si y solo si todos los elementos de $L$ tienen polinomios mínimos con raíces simples. Es decir, esta desigualdad es una igualdad \textbf{si y solo si $L/K$ es una extensión separable}.
        
        \item \textbf{La primera desigualdad ($|\operatorname{Gal}(L/K)| \le [L:K]_s$):}
        Para que el número de $K$-automorfismos sea exactamente igual al número de $K$-inmersiones en $\overline{L}$, toda inmersión $\tau: L \rightarrow \overline{L}$ debe ser, en realidad, un automorfismo de $L$. 
        Esto significa que para cualquier $\tau \in S_{K}^{L}$, se debe cumplir que $\tau(L) = L$ (o de forma equivalente, $\tau(L) \subseteq L$).
        Como vimos en el Teorema 2.11 (Condiciones equivalentes de normalidad), que todo $K$-homomorfismo en la clausura algebraica deje invariante a $L$ es cierto \textbf{si y solo si $L/K$ es una extensión normal}.
    \end{itemize}

    En conclusión, para que los extremos de la cadena (*) sean iguales, es decir, para que $|\operatorname{Gal}(L/K)| = [L:K]$, es necesario y suficiente que se den ambas igualdades intermedias. 
    Esto ocurre si y solo si la extensión $L/K$ es simultáneamente normal y separable. 
    Por definición, esto significa que $L/K$ es una extensión de Galois. Así queda probada la equivalencia $(1) \iff (2)$.

    \vspace{0.4cm}
    \textbf{Demostración de (1) $\iff$ (3):}

    Esta equivalencia es ahora una consecuencia rápida de la equivalencia anterior y de las propiedades hereditarias de las extensiones de Galois.

    \begin{itemize}
        \item \textbf{(1) $\implies$ (3):} Supongamos que $L/K$ es de Galois y sea $E$ cualquier subextensión ($K \subseteq E \subseteq L$). 
        Por el Teorema 5.10 (demostrado previamente usando levantamientos o el análisis de subextensiones cerradas), sabemos que si $L/K$ es de Galois, entonces $L/E$ también es una extensión de Galois.
        Aplicando la equivalencia recién demostrada $(1 \iff 2)$ pero a la extensión $L/E$, deducimos directamente que $[L:E] = |\operatorname{Gal}(L/E)|$. Como $E$ era arbitrario, se cumple (3).
        
        \item \textbf{(3) $\implies$ (1):} Si asumimos que $[L:E] = |\operatorname{Gal}(L/E)|$ para absolutamente toda subextensión $E \in \operatorname{Sub}(L/K)$, esta propiedad debe cumplirse en particular para la subextensión trivial $E = K$.
        Sustituyendo $E$ por $K$, obtenemos $[L:K] = |\operatorname{Gal}(L/K)|$, que es exactamente la condición (2).
        Y como hemos demostrado exhaustivamente que $(2) \implies (1)$, concluimos que $L/K$ es una extensión de Galois.
    \end{itemize}
\end{proof}

\begin{teorema}{Teorema Fundamental de la Teoría de Galois}
    Sea $L/K$ una extensión de Galois finita y sea $G=\operatorname{Gal}(L/K)$. Entonces se verifican las siguientes propiedades fundamentales:
    \begin{enumerate}
        \item La correspondencia de Galois es un \textbf{anti-isomorfismo de conjuntos ordenados} (una biyección que invierte el orden) entre $\operatorname{Sub}(L/K)$ (el conjunto de cuerpos intermedios) y $\operatorname{Sub}(G)$ (el conjunto de subgrupos de $G$).
        \item Si $X$ e $Y$ están ambos en $\operatorname{Sub}(L/K)$ o en $\operatorname{Sub}(G)$ y se cumple que $X \subseteq Y$, entonces se invierten los índices/grados:
        $$ [X^{\circ} : Y^{\circ}] = [Y : X] $$
        En particular, evaluando en los extremos se obtienen las siguientes igualdades:
        \begin{enumerate}[label=(\alph*)]
            \item Si $E\in \operatorname{Sub}(L/K)$ entonces $[L:E] = |E^{\circ}|$ y $[E:K] = [G:E^{\circ}]$.
            \item Si $H\in \operatorname{Sub}(G)$ entonces $|H| = [L:H^{\circ}]$ y $[G:H] = [H^{\circ}:K]$.
        \end{enumerate}
    \end{enumerate}
\end{teorema}

\begin{proof}
    \textbf{Demostración de (1): El Anti-isomorfismo.}
    
    Para demostrar que las aplicaciones de la correspondencia de Galois ($E \mapsto E^\circ$ y $H \mapsto H^\circ$) conforman una biyección perfecta entre los cuerpos intermedios y los subgrupos, necesitamos probar que son funciones inversas mutuas. Es decir, debemos garantizar que todo elemento coincide con su clausura: $E = E^{\circ\circ}$ para todo cuerpo intermedio y $H = H^{\circ\circ}$ para todo subgrupo.
    
    \begin{itemize}
        \item \textit{Para los cuerpos intermedios:} Por hipótesis, $L/K$ es una extensión de Galois. Como vimos en las condiciones equivalentes (Teorema 5.10), que una extensión sea de Galois implica que absolutamente \textbf{toda subextensión $E \in \operatorname{Sub}(L/K)$ es cerrada}. Por tanto, $E = E^{\circ\circ}$ siempre se cumple.
        
        \item \textit{Para los subgrupos:} Al ser $L/K$ una extensión finita (y de Galois), su grupo de Galois $G = \operatorname{Gal}(L/K)$ es un grupo finito (de orden igual a $[L:K]$). En consecuencia, cualquier subgrupo $H \in \operatorname{Sub}(G)$ es obligatoriamente un subgrupo finito. Por el Corolario 5.7, todo subgrupo finito del grupo de Galois es cerrado. Por tanto, $H = H^{\circ\circ}$ siempre se cumple.
    \end{itemize}
    
    Dado que todos los elementos de ambos conjuntos son cerrados, las aplicaciones restringen a una biyección exacta. Como ya sabíamos por la Proposición 5.3 que el operador $^\circ$ invierte las inclusiones (es antítono), esta biyección constituye un anti-isomorfismo de conjuntos ordenados.

    \vspace{0.4cm}
    \textbf{Demostración de (2): Fórmulas de índices y grados.}
    
    La igualdad general $[X^{\circ} : Y^{\circ}] = [Y : X]$ es consecuencia directa de la Proposición 5.6 (y su corolario). Dicha proposición establecía que si el elemento más pequeño ($X$) es cerrado, entonces los grados coinciden. Como acabamos de demostrar en el apartado (1) que \textbf{todos} los elementos son cerrados, la igualdad es universalmente válida para cualquier par $X \subseteq Y$.

    Vamos a deducir las fórmulas particulares (a) y (b) sustituyendo los casos extremos:

    \textit{Demostración de (a) - Para un cuerpo intermedio $E$:}
    \begin{itemize}
        \item Tomamos $X = E$ e $Y = L$. Aplicamos la fórmula general:
        $$ [E^\circ : L^\circ] = [L : E] $$
        Sabemos que el grupo de automorfismos que fijan todo $L$ es únicamente la identidad, luego $L^\circ = \{1\}$. 
        Sustituyendo: $[E^\circ : \{1\}] = [L : E]$. 
        Como el índice respecto al subgrupo trivial es el orden del grupo, concluimos que \textbf{$|E^\circ| = [L : E]$}.
        
        \item Tomamos $X = K$ e $Y = E$. Aplicamos la fórmula general:
        $$ [K^\circ : E^\circ] = [E : K] $$
        Sabemos que el grupo de automorfismos que fijan el cuerpo base $K$ es el grupo de Galois completo, luego $K^\circ = G$. 
        Sustituyendo directamente obtenemos \textbf{$[G : E^\circ] = [E : K]$}.
    \end{itemize}

    \textit{Demostración de (b) - Para un subgrupo $H$:}
    \begin{itemize}
        \item Tomamos $X = \{1\}$ e $Y = H$. Aplicamos la fórmula general:
        $$ [\{1\}^\circ : H^\circ] = [H : \{1\}] $$
        El cuerpo fijado por la identidad es todo el cuerpo superior, por lo que $\{1\}^\circ = L$. El índice $[H : \{1\}]$ es simplemente el orden del grupo $|H|$. 
        Sustituyendo, obtenemos \textbf{$[L : H^\circ] = |H|$}.
        
        \item Tomamos $X = H$ e $Y = G$. Aplicamos la fórmula general:
        $$ [H^\circ : G^\circ] = [G : H] $$
        El cuerpo fijado por todo el grupo de Galois es (por ser extensión de Galois) el cuerpo base $K$, luego $G^\circ = K$. 
        Sustituyendo directamente obtenemos \textbf{$[H^\circ : K] = [G : H]$}.
    \end{itemize}
    
    Estas deducciones completan la demostración de la estructura numérica de la correspondencia de Galois.
\end{proof}

Si $K\subseteq E\subseteq L$ es una torre de cuerpos y $\sigma\in \operatorname{Gal}(L/K)$, entonces $\operatorname{Res}_{E}^{L}(\sigma)$ denota la restricción de $\sigma$ a $E$. 
En principio $\operatorname{Res}_{E}^{L}(\sigma)$ es un $K$-homomorfismo de $E$ en $L$, pero si $E/K$ es normal entonces $\sigma\in \operatorname{Gal}(E/K)$. % 48]
Eso es lo que pasa en las condiciones de la siguiente proposición y está claro que en tal caso $\operatorname{Res}_{E}^{L}:\operatorname{Gal}(L/K)\rightarrow \operatorname{Gal}(E/K)$ es un homomorfismo de grupos. % 49]

\begin{proposicion}{Condiciones equivalentes para una extensión de Galois} % 50]
    Sea $L/K$ una extensión finita de Galois. Si $E\in \operatorname{Sub}(L/K)$ entonces las siguientes condiciones son equivalentes: % 50]
    \begin{enumerate}[label=(1)]
        \item $E/K$ es de Galois. % 51]
        \item $E/K$ es normal. % 52]
        \item $\sigma(E)\subseteq E$ para todo $\sigma\in \operatorname{Gal}(L/K)$. % 53]
        \item $\operatorname{Gal}(L/E)$ es normal en $\operatorname{Gal}(L/K)$. % 54]
    \end{enumerate}
    Además, si estas condiciones se satisfacen, entonces la aplicación de restricción % 55]
    $$\operatorname{Res}_{E}^{L}:\operatorname{Gal}(L/K)\rightarrow \operatorname{Gal}(E/K), \quad \sigma\mapsto\sigma|_{E}$$ % 56]
    es suprayectiva y como su núcleo es $\operatorname{Gal}(L/E)$, se tiene que % 5
    $$\operatorname{Gal}(E/K)\simeq\frac{\operatorname{Gal}(L/K)}{\operatorname{Gal}(L/E)}$$ % 58]
\end{proposicion}

\begin{proof}
    Vamos a desglosar el ciclo de implicaciones y la demostración del isomorfismo final paso a paso. Recordemos que por hipótesis global, $L/K$ es una extensión de Galois (finita, normal y separable).

    \vspace{0.3cm}
    \textbf{(1) $\iff$ (2): La separabilidad se hereda (es gratis).} \\
    Recordemos que una extensión es de Galois si y solo si es normal y separable. 
    Como $L/K$ es separable por hipótesis, y la separabilidad es una propiedad que se hereda a las subextensiones (Proposición 4.13), la extensión intermedia $E/K$ es automáticamente separable. 
    Por tanto, para que $E/K$ sea de Galois, la única condición que le falta cumplir es ser normal. Esto hace que las afirmaciones (1) y (2) sean estrictamente equivalentes.

    \vspace{0.3cm}
    \textbf{(2) $\implies$ (3): La normalidad encierra a las raíces conjugadas.} \\
    Supongamos que $E/K$ es normal. Tomemos un elemento cualquiera $\alpha \in E$ y un automorfismo cualquiera $\sigma \in \operatorname{Gal}(L/K)$. Nuestro objetivo es demostrar que $\sigma(\alpha)$ se queda dentro de $E$.
    
    Sea $p = \operatorname{Min}_{K}(\alpha)$ su polinomio mínimo sobre $K$. Como $\alpha \in E$ y $E/K$ es normal, este polinomio factoriza completamente dentro de $E$. Es decir, todas sus raíces $\alpha_1, \alpha_2, \dots, \alpha_n$ pertenecen a $E$.
    
    Ahora, apliquemos el automorfismo $\sigma$ a $\alpha$. Como $\sigma$ es un $K$-automorfismo, deja fijos los coeficientes de $p$ (que están en $K$), lo que implica por el Lema de Invarianza que $\sigma$ envía raíces de $p$ a raíces de $p$.
    Por consiguiente, $\sigma(\alpha)$ debe ser obligatoriamente una de esas raíces $\alpha_i$.
    Y como ya sabíamos que todas las $\alpha_i \in E$, concluimos que $\sigma(\alpha) \in E$. Como esto vale para todo $\alpha \in E$, hemos demostrado que $\sigma(E) \subseteq E$.

    \vspace{0.3cm}
    \textbf{(3) $\implies$ (4): El subgrupo de Galois es normal.} \\
    Supongamos que $\sigma(E) \subseteq E$ para todo $\sigma \in \operatorname{Gal}(L/K)$. Queremos ver que el subgrupo $H = \operatorname{Gal}(L/E)$ es un subgrupo normal del grupo total $G = \operatorname{Gal}(L/K)$.
    
    Por la definición de subgrupo normal en teoría de grupos, debemos demostrar que para cualquier $\sigma \in G$ y cualquier $\tau \in H$, el elemento conjugado $\sigma^{-1}\tau\sigma$ pertenece a $H$.
    ¿Qué significa pertenecer a $H = \operatorname{Gal}(L/E)$? Significa ser un automorfismo de $L$ que deja fijo a todo elemento de $E$.
    
    Tomemos un elemento cualquiera $\alpha \in E$ y evaluemos la composición:
    \begin{itemize}
        \item Primero actúa $\sigma$: Por la hipótesis (3), sabemos que $\sigma(\alpha) \in E$. Llamemos $y = \sigma(\alpha)$, con $y \in E$.
        \item Luego actúa $\tau$: Al ser $\tau \in \operatorname{Gal}(L/E)$, deja fijos todos los elementos de $E$. Como $y \in E$, entonces $\tau(y) = y$. Sustituyendo de vuelta: $\tau(\sigma(\alpha)) = \sigma(\alpha)$.
        \item Finalmente actúa $\sigma^{-1}$: Aplicamos $\sigma^{-1}$ a ambos lados de la igualdad anterior:
        $$ \sigma^{-1}(\tau(\sigma(\alpha))) = \sigma^{-1}(\sigma(\alpha)) = \alpha $$
    \end{itemize}
    Hemos demostrado que el automorfismo conjugado $\sigma^{-1}\tau\sigma$ deja fijo el elemento $\alpha$. Como esto es válido para todo $\alpha \in E$, el conjugado pertenece a $\operatorname{Gal}(L/E)$, probando que es un subgrupo normal.

    \vspace{0.3cm}
    \textbf{(4) $\implies$ (2): El truco del cuerpo fijo.} \\
    Supongamos que $\operatorname{Gal}(L/E)$ es un subgrupo normal de $\operatorname{Gal}(L/K)$. Queremos ver que $E/K$ es normal.
    Usaremos el criterio de que todo $K$-homomorfismo $\rho: E \rightarrow \overline{L}$ debe cumplir $\rho(E) \subseteq E$.
    
    \begin{itemize}
        \item \textit{Paso 1: Extender el homomorfismo.} Como $L/E$ es una extensión algebraica, el Teorema de Extensión nos permite prolongar $\rho$ a un $K$-homomorfismo global $\sigma: L \rightarrow \overline{L}$.
        \item \textit{Paso 2: $\sigma$ es un automorfismo.} Como la extensión total $L/K$ es normal por hipótesis, cualquier $K$-homomorfismo de $L$ cumple $\sigma(L) = L$. Esto nos asegura que $\sigma \in \operatorname{Gal}(L/K)$.
        \item \textit{Paso 3: Bajar a E.} Sea $\alpha \in E$. Queremos probar que $\rho(\alpha) = \sigma(\alpha) \in E$.
        Sabemos que la subextensión $L/E$ es de Galois (por ser $L/K$ de Galois), luego por el Teorema Fundamental, $E$ es exactamente el cuerpo fijo de su grupo de Galois: $E = \operatorname{Gal}(L/E)^\circ$.
        Para probar que $\sigma(\alpha) \in E$, basta con ver que es invariante bajo cualquier $\tau \in \operatorname{Gal}(L/E)$.
        
        Evaluemos $\tau(\sigma(\alpha))$. Esto es equivalente a $\sigma(\sigma^{-1}\tau\sigma(\alpha))$.
        Por hipótesis, $\operatorname{Gal}(L/E)$ es normal, luego el conjugado $\sigma^{-1}\tau\sigma$ es algún elemento $\tau' \in \operatorname{Gal}(L/E)$.
        Como $\alpha \in E$ y $\tau'$ fija $E$, tenemos que $\tau'(\alpha) = \alpha$.
        Sustituyendo: $\tau(\sigma(\alpha)) = \sigma(\tau'(\alpha)) = \sigma(\alpha)$.
        
        Como $\sigma(\alpha)$ queda fijo ante todo $\tau \in \operatorname{Gal}(L/E)$, obligatoriamente $\sigma(\alpha) \in \operatorname{Gal}(L/E)^\circ = E$.
    \end{itemize}
    Esto demuestra que todo $\rho(E) \subseteq E$, por lo que $E/K$ es normal.

    \vspace{0.3cm}
    \textbf{Demostración de la aplicación de Restricción y el Isomorfismo:} \\
    Supongamos ahora que se verifican las condiciones. Definimos la aplicación de restricción:
    $$f = \operatorname{Res}_{E}^{L}: \operatorname{Gal}(L/K) \rightarrow \operatorname{Gal}(E/K), \quad \sigma \mapsto \sigma|_{E}$$
    \begin{itemize}
        \item \textit{Buena definición:} La aplicación tiene sentido porque por la condición (3), $\sigma(E) \subseteq E$. Al ser $\sigma$ una inyección lineal en un espacio de dimensión finita sobre $K$, forzosamente $\sigma(E) = E$. Luego la restricción de $\sigma$ a $E$ es efectivamente un $K$-automorfismo de $E$, es decir, pertenece a $\operatorname{Gal}(E/K)$.
        \item \textit{Es homomorfismo:} La restricción de una composición es la composición de las restricciones.
        \item \textit{El Núcleo:} ¿Qué elementos $\sigma \in \operatorname{Gal}(L/K)$ van a parar al elemento neutro de $\operatorname{Gal}(E/K)$ (que es la identidad $\operatorname{id}_E$)? Exactamente aquellos que cumplen $\sigma|_E = \operatorname{id}_E$, es decir, los que dejan fijos todos los elementos de $E$. Por definición, este conjunto es $\operatorname{Gal}(L/E)$. Por tanto, $\operatorname{Ker}(f) = \operatorname{Gal}(L/E)$.
    \end{itemize}

    Aplicamos el \textbf{Primer Teorema de Isomorfía} de grupos:
    $$ \frac{\operatorname{Gal}(L/K)}{\operatorname{Ker}(f)} \simeq \operatorname{Im}(f) \subseteq \operatorname{Gal}(E/K) $$
    Para ver que la imagen es todo $\operatorname{Gal}(E/K)$ (es decir, que $f$ es suprayectiva), comparamos las cardinalidades. 
    Como las tres extensiones implicadas ($L/K$, $L/E$ y $E/K$) son de Galois, el orden de sus grupos de Galois coincide exactamente con el grado de sus extensiones.
    
    $$|\operatorname{Im} f| = \frac{|\operatorname{Gal}(L/K)|}{|\operatorname{Gal}(L/E)|} = \frac{[L:K]}{[L:E]}$$
    Por la propiedad multiplicativa del grado en la torre $K \subseteq E \subseteq L$, sabemos que $[L:K] = [L:E][E:K]$, luego la fracción se simplifica a:
    $$|\operatorname{Im} f| = [E:K]$$
    Y como $E/K$ es de Galois, $[E:K] = |\operatorname{Gal}(E/K)|$.
    
    Al tener la imagen el mismo tamaño finito que el codominio, la aplicación $f$ es suprayectiva, y el isomorfismo queda establecido:
    $$\operatorname{Gal}(E/K) \simeq \frac{\operatorname{Gal}(L/K)}{\operatorname{Gal}(L/E)}$$
\end{proof}
En el siguiente Teorema volvemos a encontrar una versión diferente de homomorfismo de restricción.

\begin{teorema}{Teorema de las Irracionalidades Accesorias de Lagrange}
    Sean $L/K$ y $E/K$ dos extensiones admisibles y supongamos que la primera es finita y de Galois. Entonces $LE/E$ y $L/L\cap E$ son extensiones de Galois finitas y el homomorfismo de restricción
    $$\operatorname{Res}_{L}^{LE}:\operatorname{Gal}(LE/E)\rightarrow \operatorname{Gal}(L/L\cap E)$$
    es un isomorfismo de grupos.
\end{teorema}

\begin{proof}
    Vamos a dividir la demostración en cuatro bloques lógicos para garantizar el rigor en cada afirmación.

    \vspace{0.3cm}
    \textbf{Paso 1: Naturaleza de las extensiones (Finitud y propiedad de Galois)}
    \begin{itemize}
        \item \textit{La extensión $L/(L\cap E)$:} Como $L/K$ es de Galois por hipótesis, sabemos por el Teorema 5.10 (propiedad hereditaria) que cualquier subextensión superior es de Galois. Al ser $L\cap E$ un cuerpo intermedio ($K \subseteq L\cap E \subseteq L$), la extensión $L/(L\cap E)$ hereda automáticamente la finitud y la propiedad de Galois.
        \item \textit{La extensión $LE/E$:} Por hipótesis, $L/K$ es finita y de Galois. Esto significa que $L$ es el cuerpo de descomposición de un cierto polinomio separable $p \in K[X]$. 
        Al considerar el cuerpo compuesto $LE$, podemos ver este mismo polinomio $p$ como un elemento de $E[X]$ (ya que $K \subseteq E$). El cuerpo $LE$ se forma adjuntando a $E$ las raíces de $p$, por lo que $LE$ es exactamente el cuerpo de descomposición de $p$ sobre $E$. Al ser un polinomio separable, $LE/E$ es una extensión normal y separable, es decir, de Galois. Su finitud se deduce de la Proposición 1.18 (el grado del compuesto no supera el producto de los grados).
    \end{itemize}

    \vspace{0.3cm}
    \textbf{Paso 2: Buena definición de la aplicación de restricción} \\
    Definimos la aplicación $f = \operatorname{Res}_{L}^{LE}$, que toma un automorfismo $\sigma \in \operatorname{Gal}(LE/E)$ y lo restringe al cuerpo $L$, denotándolo $\sigma|_L$. ¿Por qué $\sigma|_L \in \operatorname{Gal}(L/L\cap E)$?
    \begin{itemize}
        \item Como $L/K$ es normal, sabemos que cualquier $K$-homomorfismo del cuerpo compuesto que se aplique sobre $L$ debe cumplir que la imagen de $L$ es $L$. Como $\sigma$ fija $E$ (y por ende fija $K \subseteq E$), $\sigma$ es un $K$-homomorfismo, luego $\sigma(L) = L$. Esto hace que $\sigma|_L$ sea un automorfismo bien definido de $L$.
        \item Además, como $\sigma \in \operatorname{Gal}(LE/E)$, $\sigma$ deja fijos \textbf{todos} los elementos de $E$. En particular, dejará fijos los elementos de la intersección $L \cap E$. Por tanto, $\sigma|_L$ es un automorfismo de $L$ que fija $L \cap E$, lo que demuestra que $\sigma|_L \in \operatorname{Gal}(L/L\cap E)$.
    \end{itemize}

    \vspace{0.3cm}
    \textbf{Paso 3: Inyectividad de $f$} \\
    Para demostrar que $f$ es un homomorfismo inyectivo, basta con calcular su núcleo ($\operatorname{Ker}(f)$) y ver que solo contiene al elemento neutro (la identidad $\operatorname{id}_{LE}$).
    
    Supongamos que $\sigma \in \operatorname{Ker}(f)$. Por definición, esto significa que $f(\sigma) = \sigma|_L = \operatorname{id}_L$. Es decir, $\sigma(x) = x$ para todo $x \in L$.
    Pero recordemos de dónde viene $\sigma$: es un elemento de $\operatorname{Gal}(LE/E)$, por lo que, por definición, también fija todo elemento de $E$.
    
    El cuerpo compuesto $LE$ está generado por las sumas, productos y cocientes de elementos de $L$ y de $E$. Si $\sigma$ deja fijos todos los elementos de $L$ y todos los elementos de $E$, obligatoriamente deja fijos todos los elementos generados por ellos. Por tanto, $\sigma(x) = x$ para todo $x \in LE$, lo que implica que $\sigma = \operatorname{id}_{LE}$. 
    Como $\operatorname{Ker}(f) = \{\operatorname{id}_{LE}\}$, la aplicación es inyectiva.

    \vspace{0.3cm}
    \textbf{Paso 4: Suprayectividad de $f$ (El argumento del cuerpo fijo)} \\
    Este es el paso más brillante de la demostración. Denotemos la imagen de nuestra aplicación como $H = \operatorname{Im} f \subseteq \operatorname{Gal}(L/L\cap E)$. 
    Queremos demostrar que $H$ es todo el grupo, lo cual haremos utilizando la correspondencia de Galois y demostrando que el cuerpo fijo de $H$ es exactamente $L \cap E$.

    Denotemos el cuerpo fijo de $H$ (dentro de $L$) como $H^\circ = \{ \alpha \in L \mid \tau(\alpha) = \alpha, \, \forall \tau \in H \}$.
    
    \begin{itemize}
        \item \textit{Inclusión trivial ($L\cap E \subseteq H^\circ$):} Como $H$ es un subgrupo de $\operatorname{Gal}(L/L\cap E)$, todos los automorfismos de $H$ fijan por definición a $L \cap E$. Por tanto, $L \cap E$ está contenido en el cuerpo fijo de $H$.
        
        \item \textit{Inclusión profunda ($H^\circ \subseteq L\cap E$):} Sea $\alpha$ un elemento cualquiera del cuerpo fijo, $\alpha \in H^\circ$.
        Para cualquier automorfismo $\sigma \in \operatorname{Gal}(LE/E)$, su imagen $f(\sigma)$ pertenece a $H$. 
        Por estar $\alpha$ en el cuerpo fijo de $H$, sabemos que:
        $$ f(\sigma)(\alpha) = \alpha $$
        Pero por la propia definición de la restricción, $f(\sigma)(\alpha)$ es simplemente $\sigma(\alpha)$. Por tanto:
        $$ \sigma(\alpha) = \alpha \quad \forall \sigma \in \operatorname{Gal}(LE/E) $$
        Esta última afirmación significa que $\alpha$ es un elemento de $LE$ que queda invariante bajo \textbf{todo} automorfismo de $\operatorname{Gal}(LE/E)$. Como sabemos que la extensión $LE/E$ es de Galois, el único cuerpo fijo de su grupo de Galois es el cuerpo base $E$. Es decir, $\operatorname{Gal}(LE/E)^\circ = E$.
        Concluimos que $\alpha \in E$.
        
        Pero, al principio, habíamos tomado $\alpha \in H^\circ \subseteq L$, lo que significa que $\alpha$ también pertenece a $L$. 
        Si $\alpha \in E$ y $\alpha \in L$, entonces irremediablemente $\alpha \in L \cap E$.
        Como esto se cumple para todo $\alpha \in H^\circ$, deducimos que $H^\circ \subseteq L \cap E$.
    \end{itemize}

    Por la doble inclusión, hemos demostrado que el cuerpo fijo del subgrupo imagen es exactamente la intersección: $H^\circ = L \cap E$.
    
    Finalmente, invocamos el Teorema Fundamental de la Teoría de Galois sobre la extensión $L/(L\cap E)$. Como todo subgrupo es cerrado, podemos recuperar el subgrupo original aplicando la clausura:
    $$ H = H^{\circ\circ} = (L \cap E)^\circ = \operatorname{Gal}(L/L\cap E) $$
    
    Esto demuestra de forma concluyente que la imagen $H$ cubre todo el grupo codominio, haciendo que la aplicación $f$ sea suprayectiva y, en definitiva, un isomorfismo.
\end{proof}


\subsection{Ejemplo tocho}
    Vamos a calcular los subcuerpos del cuerpo de escisión $F$ del polinomio $X^{5}-p$, donde $p$ es un número primo, y cuáles son normales sobre $\mathbb{Q}$.
    
    Los subcuerpos de $F$ son precisamente las subextensiones de $F/\mathbb{Q}$.
     Por el Teorema Fundamental de la Teoría de Galois (Teorema 5.12) dichos cuerpos están en correspondencia biunívoca con los subgrupos de $G=\operatorname{Gal}(F/\mathbb{Q})$ y los normales son los que corresponden con subgrupos normales de $G$. Sea $\alpha=\sqrt[5]{p}$.
    
    Entonces $F=\mathbb{Q}(\alpha,\zeta_{5})$. Además, $[\mathbb{Q}(\alpha):\mathbb{Q}]=5$ y $[\mathbb{Q}(\zeta_{5}):\mathbb{Q}]=\varphi(5)=4$. Por tanto $[F:\mathbb{Q}]$ es al menos 20. Por otro lado $[F:\mathbb{Q}(\alpha)]\le[\mathbb{Q}(\zeta_{5}):\mathbb{Q}]=4$ y por tanto $[F:\mathbb{Q}]=[F:\mathbb{Q}(\alpha)][\mathbb{Q}(\alpha):\mathbb{Q}]\le 20$. 
    Luego $|G|=[F:\mathbb{Q}]=20$. Además $G$ contiene a $\operatorname{Gal}(F/\mathbb{Q}(\alpha))$ y a $\operatorname{Gal}(F/\mathbb{Q}(\zeta_{5}))$ que serán dos subgrupos de órdenes 4 y 5 respectivamente.
    
    Por el Teorema de las Irracionalidades Accesorias (Teorema 5.14), $\operatorname{Res}_{\mathbb{Q}(\zeta_{5})}^{F}:\operatorname{Gal}(F/\mathbb{Q}(\alpha))\rightarrow \operatorname{Gal}(\mathbb{Q}(\zeta_{5})/\mathbb{Q})$ es un isomorfismo8. Usando el isomorfismo entre $\operatorname{Gal}(\mathbb{Q}(\zeta_{5})/\mathbb{Q})$ y $\mathbb{Z}_{5}^{*}$ (Problema (3.15)) deducimos que $\operatorname{Gal}(F/\mathbb{Q}(\alpha))=\langle\tau\rangle$ con $\tau(\zeta_{5})=\zeta_{5}^{2}$. 
    Por otro lado tenemos otro elemento $\sigma\in \operatorname{Gal}(F/\mathbb{Q}(\zeta_{5}))$ con $\sigma(\alpha)=\zeta_{5}\alpha$ y claramente $\operatorname{Gal}(F/\mathbb{Q}(\zeta_{5}))=\langle\sigma\rangle$.
    
    Por otro lado, como $\mathbb{Q}(\zeta_{5})/\mathbb{Q}$ es de Galois pero $\mathbb{Q}(\alpha)/\mathbb{Q}$ no lo es, deducimos que $\langle\sigma\rangle$ es normal en $G$ pero $\langle\tau\rangle$ no es normal en $G$. Por tanto $\tau\sigma\tau^{-1}=\sigma^{i}$ para algún $i\in\{2,3,4\}$. 
    De hecho $i=2$ pues $\tau\sigma\tau^{-1}(\alpha)=\tau\sigma(\alpha)=\tau(\zeta_{5}\alpha)=\zeta_{5}^{2}\alpha=\sigma^{2}(\alpha)$. Por tanto, todos los elementos de $G$ tienen una única forma $\sigma^{i}\tau^{j}$ con $0\le i\le 4$ y $0\le j\le 3$.
    
    Vamos a calcular los subgrupos cíclicos. Ya tenemos tres: $1$, $\langle\sigma\rangle$ y $\langle\tau\rangle$, que tienen orden 1, 5 y 4, respectivamente. Todos los elementos de la forma $\sigma^{i}$ generan $\langle\sigma\rangle$ y $\tau$ y $\tau^{-1}$ generan $\langle\tau\rangle$. Otro subgrupo cíclico más será $\langle\tau^{2}\rangle$. 
    Este último tiene orden 2. Solo nos falta calcular los subgrupos cíclicos generados por los elementos de la forma $\sigma^{i}\tau^{j}$ con $1\le i\le 4$ y $1\le j\le 3$.
    
    Comenzamos con los de la forma $\sigma^{i}\tau^{2}$. De la igualdad $\tau\sigma=\sigma^{2}\tau$ observamos que $\tau^{2}\sigma=\sigma^{4}\tau^{2}=\sigma^{-1}\tau^{2}$. Por tanto, para cada $i$ tenemos que $(\sigma^{i}\tau^{2})^{2}=1$. O sea, cada $\sigma^{i}\tau^{2}$ tiene orden 2 .
     Esto nos proporciona cinco subgrupos de orden 2, uno de los cuales es $\langle\tau^{2}\rangle$.
    
    Por otro lado $(\sigma^{i}\tau)^{2}=\sigma^{3i}\tau^{2}$, que tiene orden 296. Por tanto, $\langle\sigma^{i}\tau\rangle$ tiene orden 4 y su único subgrupo de orden 2 es $\langle\sigma^{3i}\tau^{2}\rangle$96. 
    Como estos últimos son distintos para los cinco valores distintos de $i$, obtenemos de esta forma cinco subgrupos cíclicos de orden 4, uno de los cuales es $\langle\tau\rangle$. Cada uno de estos subgrupos tiene dos elementos de orden 49. 
    Más concretamente $\langle\sigma^{i}\tau\rangle$ también está generado por $(\sigma^{i}\tau)^{3}=\sigma^{i}\tau\sigma^{3i}\tau^{2}=\sigma^{2i}\tau^{3}$9. Por tanto, ya tenemos todos los subgrupos cíclicos:
    \begin{itemize}
        \item De orden 1: $1$.
        \item De orden 2: $\langle\sigma^{i}\tau^{2}\rangle$ con $0\le i\le 4$.
        \item De orden 4: $\langle\sigma^{i}\tau\rangle$ con $0\le i\le 4$.
        \item De orden 5: $\langle\sigma\rangle$.
    \end{itemize}
    
    Calculamos ahora los grupos generados por dos elementos $g$ y $h$. Por supuesto, si uno de ellos está en el subgrupo generado por el otro lo que obtendremos es uno de los grupos cíclicos, con lo que suponemos que $g\notin\langle h\rangle$ y $h\notin\langle g\rangle$.
    
    Supongamos primero que uno de los dos tiene orden 5. Por ejemplo, supongamos que $|g|=5$ y por tanto $h$ tiene orden 2 ó 4. Si $h$ tiene orden 4 entonces $\langle g,h\rangle=G$. Sin embargo si $h$ tiene orden 2 entonces $\langle g,h\rangle=\langle\sigma,\tau^{2}\rangle$ y como $\tau^{2}\sigma=\sigma^{-1}\tau^{2}$ tenemos que $\langle\sigma,\tau^{2}\rangle$ tiene orden 10. 
    
    En los demás casos $g$ y $h$ tienen orden 2 ó 4 y vamos a ver que siempre $\langle g,h\rangle=G$. Si $|g|=|h|=4$ entonces podemos suponer que $g=\sigma^{i}\tau$ y $h=\sigma^{j}\tau$ con $i\ne j$. Por tanto, $\langle g,h\rangle$ contiene a $gh^{-1}=\sigma^{i-j}$. Como este elemento genera a $\langle\sigma\rangle$, tenemos que $\langle g,h\rangle=\langle\sigma,\tau\rangle=G$. 
    El mismo argumento muestra que si $g$ y $h$ tienen orden 2 y son distintos, entonces generan $G$. Finalmente si uno tiene orden 2 y el otro 4, por ejemplo $h$, entonces $\langle g,h\rangle$ contiene a $\langle g,h^{2}\rangle$ con $g$ y $h^{2}$ distintos de orden 2 y de nuevo obtenemos que $\langle g,h\rangle=G$.


    El retículo de subgrupos que obtenemos es el siguiente: 
    \begin{figure}
    \centering
    \includegraphics[width=1.3\linewidth]{imagenes/Diagrama1T5.png}
    \end{figure}



    Dando la vuelta al diagrama obtenemos las inclusiones entre los subcuerpos de $F$. Pero antes de hacerlo vamos a calcular los cuerpos. 
    
    Claramente $G^{\circ}=\mathbb{Q}$, $1^{\circ}=F$, $\langle\sigma\rangle^{\circ}=\mathbb{Q}(\zeta_5)$, $\langle\tau\rangle^{\circ}=\mathbb{Q}(\alpha)$. 
    Cada uno de los $\langle\sigma^i\tau\rangle^{\circ}$ tiene que tener grado 5 sobre $\mathbb{Q}$ y serán los únicos subcuerpos de grado 5 sobre $\mathbb{Q}$, que necesariamente son $\mathbb{Q}(\alpha)$, $\mathbb{Q}(\zeta_5\alpha)$, $\mathbb{Q}(\zeta_5^2\alpha)$, $\mathbb{Q}(\zeta_5^3\alpha)$ y $\mathbb{Q}(\zeta_5^4\alpha)$. % [cite: 72]
    Observamos que $\sigma^i\tau(\zeta_5^{-i}\alpha)=\sigma^i(\zeta_5^{-2i}\alpha)=\zeta_5^{-i}\alpha$. 
    Por tanto $\langle\sigma^i\tau\rangle^{\circ} = \mathbb{Q}(\zeta_5^{-i}\alpha)$. 
    
    Por otro lado $\langle\sigma,\tau^2\rangle$ es el único subgrupo que tiene grado 2 sobre $\mathbb{Q}$, cuyo cuerpo fijo además está contenido en $\langle\sigma\rangle^{\circ}=\mathbb{Q}(\zeta_5)$. 
    Observando que $\beta=\zeta_5+\zeta_5^{-1}=2\cos(2\pi/5)\in\mathbb{R}$, tenemos que $\mathbb{Q}(\beta)$ está contenido en $\mathbb{Q}(\zeta_5)$. 
    Además, como $1+\zeta_5+\zeta_5^2+\zeta_5^3+\zeta_5^4=0$ tenemos que $\beta^2=\zeta_5^2+\zeta_5^{-2}+2=-\beta+1$, con lo que $\beta$ es raíz de $X^2+X-1$, de donde $\beta=\frac{-1+\sqrt{5}}{2}$ y por tanto $\mathbb{Q}(\beta)=\mathbb{Q}(\sqrt{5})$. % [cite: 7
    


    Por tanto $\langle\sigma,\tau^2\rangle^{\circ}=\mathbb{Q}(\sqrt{5})$.
    Ahora observamos que $\langle\sigma^{3i}\tau^2\rangle=\langle\sigma,\tau^2\rangle\cap\langle\sigma^i\tau\rangle$.
    Como la correspondencia de Galois es un anti-isomorfismo de retículos deducimos que $\langle\sigma^{3i}\tau^2\rangle^{\circ}=\langle\sigma,\tau^2\rangle^{\circ}\langle\sigma^i\tau\rangle^{\circ}=\mathbb{Q}(\sqrt{5},\zeta_5^{-i}\alpha)$. % [cite: 78]
    
    Por tanto el retículo de subcuerpos es el siguiente: 
    
    \begin{figure}
    \centering
    \includegraphics[width=1.3\linewidth]{imagenes/Diagrama2T5.png}
    \end{figure}
    Obsérvese que los únicos subgrupos normales de $G$ son $1$, $\langle\sigma\rangle$, $\langle\sigma,\tau^2\rangle$ y $G$.
    Por tanto, los únicos subcuerpos de $F$ que son normales sobre $\mathbb{Q}$ son $\mathbb{Q}$, $\mathbb{Q}(\sqrt{5})$, $\mathbb{Q}(\zeta_5)$ y $F$. 

